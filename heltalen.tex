\chapter{De hela talen}
\label{ch:Heltalen}
\lettrine{V}{i är nu välbekanta} med de naturliga talen,
\(\N=\{0,1,2,\ldots\}\).
Vi har operationerna addition och multiplikation och vet hur dessa fungerar.
I detta kapitel kommer vi att utöka de naturliga talen med avseende på
addition.

Om vi adderar två tal \(a\) och \(b\) får vi ett tredje tal \(c\), vi skriver
detta som \(a+b=c\).
Det finns dock inget sätt att ta oss tillbaka till \(a\), det vill säga det
finns inget naturligt tal \(d\) sådant att \(c+d=a+b+d=a\).
Ett annat sätt att säga det på är att det finns inga naturliga tal \(n\) och
\(m\) sådana att deras summa \(n+m=0\) är lika med identitetselementet noll.
Detta är dock en väldigt intressant och önskvärd egenskap.
Vi ska förtydliga vad vi menar och ge denna egenskap ett namn.
\begin{definition}\label{def:invers}\index{invers}
  Givet en mängd \(M\) med en definierad binär operation \(\diamond\colon 
  M\times M\to M\) och ett identitetselement \(e\).
  Ett element \(b\) kallas för \emph{inversen till} ett element \(a\) om
  \(a\diamond b = b\diamond a = e\).
\end{definition}
\begin{exercise}
  Utforska inversbegreppet.
  Frågor att inspirera:
  I \cref{def:invers}, är \(a\) inversen till något element?
  Hur många inverser kan ett element ha?
  Har alla element en invers?
\end{exercise}
%\begin{remark}
%  Av inversens natur följer att om \(b\) är inversen till \(a\), då är även
%  \(a\) inversen till \(b\).
%\end{remark}

Vi ska i detta kapitel tillföra denna egenskap till additionen för de naturliga
talen.
Resultatet är vad som kallas de hela talen, och mängden av de hela talen
betecknas\footnote{Anledningen till att de betecknas med \(\Z\) är tyskans
\emph{Zahlen} som betyder tal.} \(\Z\).


%%%%%%%%%%%%%%%%%%%%%%%%%%%%%%%%%%%
% UTÖKNINGEN
%%%%%%%%%%%%%%%%%%%%%%%%%%%%%%%%%%%
\section{Utökningen av de naturliga talen}
\label{sec:HeltalensKonstruktion}
Låt oss nu skapa de hela talen.
Det enda vi har att tillgå är de naturliga talen, dessa vet vi att de
existerar och hur de fungerar.
Vi börjar med att låta mängden \(M = \N\times\N =
\{(a,b)\colon a\text{ och }b\text{ är naturliga tal}\}\) vara mängden av alla
ordnade par av naturliga tal.
\begin{example}
  Vi har exempelvis att \((1,2)\in M\) och \((2,1)\in M\) samt att
  \((1,2)\neq (2,1)\).
\end{example}

Låt oss nu införa en relation \(\sim\) på denna mängd.
Om \((a,b)\) och \((c,d)\) är ordnade par av naturliga tal, det vill säga
element i mängden \(M\), då vill vi att \((a,b)\sim (c,d)\) ska vara sant
precis när \(a+d=b+c\).
Denna relation \(\sim\) är faktiskt en ekvivalensrelation.
För att visa detta kollar vi att relationen uppfyller axiomen i
\cref{def:Ekvivalensrelation} för en ekvivalensrelation.
Additionen är kommutativ och eftersom att \(a+b=b+a\) har vi
\((a,b)\sim (a,b)\), och följaktligen är den reflexiv.
Om \((a,b)\sim (c,d)\) då är \(a+d=b+c\) och således 
\begin{align*}
  c+b &= b+c,\\
  d+a &= a+d \text{\ och\ } \\
  c+b &= d+a.
\end{align*}
Den sista ekvationen ger precis \((c,d)\sim (a,b)\).
Relationen måste då vara symmetrisk.
För att visa transitiviteten behövs följande resultat.
\begin{lemma}\label{lem:AdditionInjektiv}
  Låt \(x\) och \(y\) vara naturliga tal.
  Om \(z\) är ett naturligt tal och \(x+z=y+z\) då är \(x=y\).
\end{lemma}
\begin{exercise}
  Bevisa \cref{lem:AdditionInjektiv}.
\end{exercise}
Om \((a,b)\sim (c,d)\) och \((c,d)\sim (e,f)\), då har vi först att
\begin{equation}
  \label{eq:VisaHeltalensEkvivalensrelation1}
  a+d = b+c
\end{equation}
och sedan att
\begin{equation}
  \label{eq:VisaHeltalensEkvivalensrelation2}
  c+f=d+e.
\end{equation}
Detta ger
\begin{equation*}
  a+d+f = b+c+f = b+d+e.
\end{equation*}
Eftersom att additionen är kommutativ får vi \(a+f+d = b+e+d\) och
enligt \cref{lem:AdditionInjektiv} har vi då att \(a+f = b+e\) som
betyder att \((a,b)\sim (e,f)\).
Eftersom att relationen då även är transitiv är relationen en
ekvivalensrelation.

Vi betecknar ekvivalensklasserna på följande sätt:
Om \((a,b)\) är ett element i \(M\), då är
\([(a,b)]=\{(x,y)\in M\colon (a,b)\sim (x,y)\}\).
Det vill säga, \([(a,b)]\) innehåller alla talpar i \(M\) som uppfyller
ekvivalensrelationen med talparet \((a,b)\).
\begin{example}
  Vi har att \((0,0)\sim (1,1)\) eftersom att \(0+1 = 0+1\).
  Följaktligen har vi \([(0,0)]=[(1,1)]\) och \((1,1)\in [(0,0)]\), men
  naturligtvis även \((0,0)\in [(0,0)]\).
\end{example}
\begin{remark}
  Kom ihåg att det inte spelar någon roll om vi väljer \((0,0)\) eller
  \((1,1)\) som representant för
  ekvivalensklassen \([(0,0)]=[(1,1)]\) eftersom att det är samma
  ekvivalensklass.
\end{remark}
\begin{example}
  Vi har däremot \((0,0)\nsim (1,0)\) eftersom att \(0+0\neq 0+1\),
  och följaktligen \((1,0)\notin [(0,0)]\).
\end{example}

Låt oss nu definiera en binär operation \(+\) som vi kallar för addition och en
binär operation \(\cdot\) som vi kallar för multiplikation på mängden av
ekvivalensklasser hos \(M\).
Vi låter \([(a,b)]+[(c,d)] = [(a+c, b+d)]\) och \([(a,b)]\cdot [(c,d)] =
[(ac+bd, ad+bc)]\).
Vi måste dock visa att dessa är väldefinierade.
%\begin{proof}[Väldefinierad addition]
Om \((a,b)\sim (a^\prime, b^\prime)\) och \((c,d)\sim (c^\prime,
d^\prime)\), då vill vi visa att \((a+c, b+d)\sim (a^\prime+c^\prime,
b^\prime+d^\prime)\).
Vi har från \((a,b)\sim (a^\prime,b^\prime)\) att \(a+b^\prime =
b+a^\prime\) och från \((c,d)\sim (c^\prime,d^\prime)\) att
\(c+d^\prime=d+c^\prime\).
Om vi adderar vänsterleden i de båda ekvationerna, då måste detta vara lika
med summan av högerleden.
Det vill säga, \((a+b^\prime) + (c+d^\prime) = (b+a^\prime) +
(d+c^\prime)\).
Eftersom att additionen är associativ och kommutativ får vi att
\begin{equation*}
  a+c+b^\prime+d^\prime = b+d+a^\prime+c^\prime,
\end{equation*}
och således måste \((a+c,b+d)\sim (a^\prime+c^\prime, b^\prime+d^\prime)\).
%\end{proof}
\begin{exercise}
  Visa att multiplikationen också är väldefinierad.
\end{exercise}

\begin{example}\label{ex:HeltalAddition}
  Om vi adderar \([(1,2)]\) och \([(2,4)]\) får vi
  \begin{equation*}
    [(1,2)]+[(2,4)] = [(1+2,2+4)] = [(3, 6)] = [(0,3)].
  \end{equation*}
\end{example}
\begin{example}\label{ex:HeltalMultiplikation}
  Om vi multiplicerar \([(1,2)]\) och \([(2,4)]\) får vi
  \begin{equation*}
    [(1,2)]\cdot [(2,4)] = [(1\cdot 2 + 2\cdot 4, 1\cdot 4 + 2\cdot 2)] =
    [(10, 8)] = [(2,0)].
  \end{equation*}
\end{example}

Vi har nu objekt med egenskaperna att de beter sig som de naturliga talen, men
vi kan även addera dem för att få identitetselementet.
Identitetselementet för addition måste vara ekvivalensklassen \([(0,0)]\).
Eftersom att \([(a,b)]+[(0,0)] = [(a+0,b+0)] = [(a,b)]\) och
\([(0,0)]+[(a,b)] = [(0+a,0+b)] = [(a,b)]\) måste \([(0,0)]\)
vara identitetselementet för addition.

\begin{exercise}
  Hur skulle de naturliga talen kunna representeras med dessa objekt?
\end{exercise}

Den additiva inversen till \([(a,b)]\) är \([(b,a)]\) eftersom att
\([(a,b)]+[(b,a)]=[(a+b,b+a)]=[(a+b,a+b)]\).
Eftersom att \(0+(a+b)=0+(a+b)\) har vi enligt ekvivalensrelationen att
\((0,0)\sim (a+b,a+b)\) och därför måste de tillhöra samma ekvivalensklass.
Detta innebär att \([(a,b)]\) och \([(b,a)]\) måste vara varandras inverser
under addition.

\begin{exercise}
  Kan ett tal ha fler än en invers?
\end{exercise}

Vi sammanfattar detta avsnitt med följande definitioner.
Vi börjar med att definiera vad de hela talen är.
\begin{definition}[Heltalen]\index{heltal}
  Låt \(M=\N\times\N\) vara mängden av alla naturliga talpar och \(\sim\)
  vara ekvivalensrelationen på mängden \(M\) sådan att \((a,b)\sim (c,d)\)
  är uppfylld för elementen \((a,b)\) och \((c,d)\) i \(M\) om \(a+d=b+c\).

  Låt \(\Z = M/\!\sim = \{[(a,b)]\colon (a,b)\in M\}\) vara mängden av alla
  ekvivalensklasser för \(M\) under ekvivalensrelationen \(\sim\).
  Varje element \(n\) i \(\Z\) kallar vi ett \emph{heltal}.

  Låt också \(0\) beteckna \([(0,0)]\in\Z\), \(1\) beteckna \([(1,0)]\in\Z\)
  och \(a\) beteckna \([(a,0)]\in\Z\) för alla naturliga tal \(a\).
  Beteckna också den additiva inversen \([(0,a)]\in\Z\) till \([(a,0)]\) med
  \(-a\).
\end{definition}
De nya beteckningarna och ekvivalensklasserna visas i
\cref{fig:HeltalensEkvivalensklasser}.
\begin{figure}
  \centering
  \begin{asy}
    // Courtesy of Thomas Douillard, thomas.douillard at gmail.com
    // From Wikipedia
    // import settings
    import graph;
    size(10cm,0);
    // returns a pair representation of a relative number
    // of the equivalent class
    pair pairRepresentation(int n){//
            if(n>0){//
                    return(n,0);
            }else{//
                    return(0,-n);
            }
    }
    string nullString(real r){//
            return "";
    }
    void drawCoordinates(pair point, align align=NoAlign){//
            label("$("+string(point.x)+","+string(point.y)+")$",point,align,
              fontsize(10));
    }
    unitsize(50,50);
    int num = 5;
    int i;
    for (i=-1*num ; i<=num ; ++i) {//
            pair point = pairRepresentation(i) ;
            dot(point,red);
            // equivalence classes labelled with usual names, in blue
            label("$\mathbf{"+string(i)+"}$",point,5SW,fontsize(10)+blue);
            int j;
            for(j=abs(i);j<num;++j) {//
                    drawCoordinates(point,E);
                    pair nextpoint = point + (1,1);
                    draw(point -- nextpoint,blue+Dotted+linewidth(2));
                    dot(point,red+linewidth(5));
                    point=nextpoint;
            }
            dot(point,red);
            draw(point -- point+(0.5,0.5),blue+Dotted+linewidth(2));
            dot(point,red+linewidth(5));
            drawCoordinates(point,E);
    }
    // axes
    real decay=-0.2;
    ticks tick=RightTicks(N=0,n=1,end=false,nullString);
    xaxis("$n_1$",YEquals(decay),decay,num+1.0,tick,Arrow);
    yaxis("$n_2$",XEquals(decay),decay,num+1.0,LeftTicks(N=0,n=1,end=false,nullString),Arrow);
  \end{asy}
  \caption{%
    Ekvivalensklasser för par av naturliga tal \((n_1,n_2)\) under
    relationen \(\sim\).
    Bild:~\cite{Wikipedia2013Integer}.
  }\label{fig:HeltalensEkvivalensklasser}
\end{figure}

När vi nu vet vad ett heltal är kan det vara passande att införa aritmetiska
operationer för dem.
Vi börjar med additionen och fortsätter med multiplikationen senare.
\begin{definition}[Addition]\index{addition}\index{heltal!addition}
  Låt \(x=[(a,b)]\) och \(y=[(c,d)]\) vara heltal och \(+\) en binär
  operation på \(\Z\), kallad \emph{addition}, sådan att
  \(x+y=[(a,b)]+[(c,d)]\) är lika med \([(a+c,b+d)]\).
\end{definition}
\begin{exercise}
  I \cref{ex:HeltalAddition}, vilka tal adderades och vilket blev resultatet?
\end{exercise}
\begin{exercise}
  Är additionen kommutativ för de hela talen?
\end{exercise}
\begin{exercise}
  Är additionen associativ för de hela talen?
\end{exercise}

Nu när vi kan addera heltal är det lämpligt att vi tittar på hur vi kan jämföra
dem annat än med likhet.
Olikheter infördes med hjälp av additionen för de naturliga talen, det är inte
förvånande att vi gör detsamma för heltalen.
\begin{definition}[Olikhet]\index{olikhet}
  Låt \(x=[(a,b)]\) och \(y=[(c,d)]\) vara heltal.
  Vi säger att \(x\) \emph{är mindre än} \(y\), betecknat \(x<y\), om
  \(a+d<b+c\) enligt relationen \(<\) för de naturliga talen.
  Vi kan också säga att \(y\) \emph{är större än} \(x\) och skriver då
  \(y>x\).
  Vi säger att \(x\) \emph{är mindre än eller lika med} \(y\) om \(a+d\leq
  b+c\), detta betecknas \(x\leq y\).
  Vi kan också säga att \(y\) \emph{är större än eller lika med} \(x\) och
  betecknas \(y\geq x\).
\end{definition}

Nu när vi kan ordna de hela talen är den naturliga efterföljande definitionen
denna.
\begin{definition}\index{negativt tal}\index{positivt tal}
  Låt oss kalla ett tal mindre än noll för ett \emph{negativt} tal, och låt
  oss kalla ett tal större än noll för ett \emph{positivt} tal.
\end{definition}
Vi har nu infört olika namn för talen på båda sidorna om talet noll.
Notera att enligt denna definition är noll varken positivt eller negativt.

\begin{exercise}
  Hur ordnas de hela talen?
\end{exercise}

Nu till inverserna.
\begin{definition}\index{negation}\index{additiv invers}
  Om \(x=[(a,b)]\) är ett heltal, då är \(-x=-[(a,b)]=[(b,a)]\) dess
  \emph{additiva invers}.
  Denna kallas även för \emph{negationen} av \(x\).
\end{definition}
\begin{remark}
  Låt \(a\) och \(b\) vara heltal och \(-b\) den additiva inversen till
  \(b\).
  Normalt skriver vi \(a+(-b)\) som \(a-b\) för att spara in på några tecken.
  Vi benämner \(a-b\) som \emph{subtraktionen}\index{subtraktion} av \(b\)
  från \(a\).
  Ur en matematisk synvinkel är subtraktion inte en egen operation utan vi
  adderar den additiva inversen för \(b\) till \(a\).
\end{remark}

\begin{exercise}
  Visa att ett nollskilt naturligt tal är ett positivt tal.
\end{exercise}
\begin{exercise}
  Visa att den additiva inversen för ett nollskilt naturligt tal är ett
  negativt tal.
\end{exercise}
\begin{exercise}
  De naturliga talens ordning är \(0 < 1 < 2 < 3 < \ldots\), enligt relationen 
  \(<\) definierad för de naturliga talen.
  Gäller denna ordning även för relationen \(<\) definierad för de hela talen?
\end{exercise}

Den andra aritmetiska operationen, multiplikation, ger vi i den sista
definitionen.
\begin{definition}[Multiplikation]\index{multiplikation}\index{heltal!multiplikation}
  Låt \(x=[(a,b)]\) och \(y=[(c,d)]\) vara heltal och \(\cdot\) en binär
  operation på \(\Z\), kallad \emph{multiplikation}, sådan att
  \(x\cdot y=[(a,b)]\cdot [(c,d)]\) är lika med \([(ac+bd,ad+bc)]\).
  Vi skriver ofta \(xy\) istället för \(x\cdot y\).
\end{definition}
\begin{exercise}
  I \cref{ex:HeltalMultiplikation}, vilka tal multiplicerades och vilket
  blev resultatet?
\end{exercise}

Vi ska nu visa ett lemma som vi kommer att behöva i kommande avsnitt.
Lemmat talar om att produkten av ett helt tal och noll blir noll.
\begin{lemma}\label{lem:IngenNolldelare}
  För alla heltal \(a\) gäller att \(a\cdot 0 = 0\cdot a = 0\).
\end{lemma}
\begin{proof}
  Antag att \(a = [(b,c)]\) där \(b\) och \(c\) är naturliga tal.
  Vi har per definition att \(a\cdot 0 = [(b,c)]\cdot [(0,0)] =
  [(b\cdot 0 + c\cdot 0, a\cdot 0 + c\cdot 0)] = [(0,0)] = 0\).
  På samma sätt får vi att \(0\cdot a = [(0,0)]\cdot [(b,c)] = [(0\cdot b +
  0\cdot c, 0\cdot c + 0\cdot b)] = [(0,0)] = 0\).
\end{proof}

\begin{exercise}
  Är multiplikationen kommutativ för de hela talen?
\end{exercise}
\begin{exercise}
  Är multiplikationen associativ för de hela talen?
\end{exercise}
\begin{exercise}
  Är multiplikationen distributiv över additionen för de hela talen?
\end{exercise}


%%%%%%%%%%%%%%%%%%%%%%%%%%%%%%%%%%%
% ALGEBRAISKA EGENSKAPER
%%%%%%%%%%%%%%%%%%%%%%%%%%%%%%%%%%%
\section{Algebraiska egenskaper för de hela talen}
\label{sec:HeltalensAlgebraiskaEgenskaper}
Det är nu dags att sammanfatta de algebraiska egenskaperna för de
hela talen.
De hela talen bygger på de naturliga talen och ska vara en utökning av dessa.
Vi hade för avsikt att de hela talen skulle ha samma algebraiska egenskaper som
vi visat att de naturliga talen har.
Vi fortsätter därför härnäst med en sammanfattning av de algebraiska egenskaper
som de hela talen måste ha.
Dessa är desamma som för de naturliga talen förutom tillägget om att varje tal
\(a\) har en additiv invers \(-a\).

\theoremstyle{plain}
\newtheorem*{AlgebraicPropertiesIntegers}{Algebraiska egenskaper för de hela
  talen}
\begin{AlgebraicPropertiesIntegers}\label{def:HeltalenEgenskaper}
  På mängden \(\Z\) av hela tal definieras två binära operationer,
  addition (\(+\)) och multiplikation (\(\cdot\)).
  För addition gäller följande:
  \begin{description}
    \item[Kommutativitet] \(a+b=b+a\) för alla \(a,b\in\Z\).
    \item[Associtivitet] \((a+b)+c=a+(b+c)\) för alla \(a,b,c\in\Z\).
    \item[Additivt identitetselement] Det finns ett element \(0\in\Z\)
      sådant att för alla \(a\in\Z\) gäller att \(0+a = a+0 = a\).
    \item[Additiv invers] För alla \(a\in\Z\) finns ett element \(-a\in\Z\)
      sådant att \(a + (-a) = (-a) + a = 0\).
  \end{description}
  För multiplikation gäller följande:
  \begin{description}
    \item[Kommutativitet] \(a \cdot b=b \cdot a\) för alla \(a,b\in\Z\).
    \item[Associtivitet] \((a \cdot b) \cdot c=a \cdot (b \cdot c)\) för
      alla \(a,b,c\in\Z\).
    \item[Multiplikativt identitetselement] Det finns ett element
      \(1\in\Z\) sådant att för alla \(a\in\Z\) gäller att
      \(1 \cdot a = a \cdot 1 = a\).
  \end{description}
  Utöver detta gäller även
  \begin{description}
    \item[Multiplikativ distributivitet över addition]
      \(a \cdot (b+c) = (a \cdot b) + (a \cdot c)\) och
      \((b+c) \cdot a = (b \cdot a) + (c \cdot a)\) för alla hela tal
      \(a,b,c\in\Z\).
  \end{description}
\end{AlgebraicPropertiesIntegers}

För en mer detaljerad behandling av dessa egenskaper hänvisas läsaren till 
\cite{Bartle2000itr,Grillet2007aa}.



%%%%%%%%%%%%%%%%%%%%%%%%%%%%%%%%%%%%%%%%%%%%
% ALGEBRAISKA EGENSKAPER FÖR DE NEGATIVA TALEN
%%%%%%%%%%%%%%%%%%%%%%%%%%%%%%%%%%%%%%%%%%%%
\section{Algebraiska egenskaper för de negativa talen}
Vi är redan bekanta med de naturliga talen, och därmed också den
delen av de hela talen som motsvarar de naturliga talen.
Därför ska vi i detta avsnitt fokusera på de nya talen, de heltal \(a<0\) som
är mindre än noll --- eller de negativa talen.

Låt oss börja med ett av de fundamentala resultaten,
% (-1)a = -a
nämligen att ett heltal \(a\) multiplicerat med \(-1\) ger dess invers \(-a\).
Vi formulerar detta som en sats.
\begin{theorem}\label{thm:FaktoriseraNegativaTal}
  Låt \(a\) vara ett heltal.
  Då är \(a\) multiplicerat med \(-1\) inversen \(-a\) till \(a\).
  Det vill säga, \((-1)\cdot a = -a\).
\end{theorem}
\begin{proof}
  Vi har att \(1+(-1) = 0\).
  Om vi multiplicerar noll med ett tal får vi enlingt
  \cref{lem:IngenNolldelare} fortfarande noll.
  Följaktligen får vi att \(a(1+(-1))=0\) eftersom att vi då multiplicerar
  \(a\) med noll.
  Eftersom att multiplikation är distributiv över addition får vi då att
  \begin{equation*}
    a(1+(-1)) = 1\cdot a + (-1)a = a + (-1)a = 0.
  \end{equation*}
  Då måste \((-1)a\) vara inversen \(-a\) till \(a\).
\end{proof}
\begin{remark}
  Notera att detta och följande resultat även kan härledas direkt från
  talparskonstruktionen i \cref{sec:HeltalensKonstruktion} trots att vi
  i detta bevis väljer att använda oss av de algebraiska egenskaperna givna i
  \cref{sec:HeltalensAlgebraiskaEgenskaper}.
\end{remark}

% (-1)(-1) = (-1)^2 = 1
Vi fortsätter med kanske det mest häpnanväckande resultatet, som är en
följdsats till föregående.
\begin{corollary}\label{NegMultNegEqPos}
  Den additiva inversen \(-1\) till heltalet ett multiplicerad med sig
  själv är ett.
  Det vill säga \((-1)(-1) = 1\).
\end{corollary}
\begin{proof}
  Eftersom att \(-1\) är ett heltal följer det från
  \cref{thm:FaktoriseraNegativaTal} att om vi multiplicerar det med \(-1\)
  får vi dess invers.
  Inversen för \(-1\) är \(1\) och således har vi att \((-1)(-1)=1\).
\end{proof}

\begin{exercise}
  Om \(a\) och \(b\) är heltal, gäller då att \((-a)(-b) = ab\) för alla
  heltal \(a\) och \(b\)?
\end{exercise}

Låt oss avsluta med ett exempel om addition av två negativa tal.
\begin{example}
  % (-1)+(-1) = -2
  Om vi adderar \(-1\) och \(-1\), vad får vi då?
  Eftersom att vi adderar två lika termer kan vi skriva additionen som
  multiplikationen \(2\cdot (-1) = (-1)+(-1)\).
  Vi vet från \cref{thm:FaktoriseraNegativaTal} att ett heltal
  multiplicerat med \(-1\) är dess invers, då får vi \((-1)+(-1)=2\cdot
  (-1) = -2\).
\end{example}
\begin{example}
  Vi kan också beräkna summan \((-1)+(-1)\) genom att använda heltalens
  distributiva egenskap.
  Eftersom att \((-1)=(-1)\cdot 1\) och att multiplikation är distributiv
  över addition får vi att \((-1)+(-1) = (-1)(1+1) = (-1)(2) = -2\).
\end{example}
\begin{exercise}
  % (-1)(a+b) = (-a)+(-b)
  Om \(a\) och \(b\) är heltal,
  gäller generellt att \((-a)+(-b)=-(a+b)\)?
\end{exercise}

\begin{exercise}
  % (-1)(a-b) = (-1)(a+(-b)) = (-a)+b
  Utred vad \(-(a-b)\) egentligen betyder.
\end{exercise}


%%%%%%%%%%%%%%%%%%%%%%%%%%%%%%%%%%%%%%%%%%%%
% POTENSER
%%%%%%%%%%%%%%%%%%%%%%%%%%%%%%%%%%%%%%%%%%%%
\subsection{Potenser för heltalen}
Vi ska nu införa potenser för de hela talen, likt de vi införde för de
naturliga talen.
Vi definierar potenser enligt följande definition.
\begin{definition}
  Låt \(a\) vara ett heltal.
  Vi definierar att \(a^1=a\).
  Om \(a^{n}\) är definierat för något naturligt tal \(n > 0\), då är
  \(a^{n+1} = a\cdot a^n\).
  Vi utläser \(a^n\) som en \emph{\(a\)-potens med exponenten \(n\)}, eller 
  \emph{\(a\) upphöjt till \(n\)}.
\end{definition}

\begin{exercise}
  Utforska potenser för de hela talen.
  Finns det några intressanta resultat om dessa potenser?
  %Gäller resultaten vi visat för de naturliga talen?
\end{exercise}

