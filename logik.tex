\chapter{Logik och bevis}
\label{ch:Logik}
\lettrine{M}{atematiken har sin} grund i logiken.
Det är logiken som ger matematiken möjligheten till ett resonemang och
möjligheten till härledning.
Matematiken utgår från några få grundläggande antaganden, kallade \emph{axiom},
från vilka matematiska resultat härleds.
En matematiker nöjer sig således inte med att undersöka några exempel -- eller
göra experiment -- inte ens tusen- eller miljontals exempel duger.
Det är detta som skiljer matematiken från exempelvis fysiken och kemin, trots
att dessa till mycket stor omfattning använder matematiken som hjälpmedel.
Vi känner inte till alla grundläggande principer för fysiken och kemin, utan 
det är (återstoden av) dessa vi försöker att finna genom grundforskningen inom 
dessa ämnen.
Vi känner däremot till alla grundläggande principer för matematiken, detta för
att matematiken är skapad av oss -- det är vi som bestämt dessa principer.
I detta kapitel ska vi lära oss mer om dessa.


%%%%%%%%%%%%%%%%%%%%%%%%%%%%%%%%%%
% LOGIK
%%%%%%%%%%%%%%%%%%%%%%%%%%%%%%%%%%
\section{Logik}

All matematisk argumentation består av \emph{utsagor}.
Utsagor\index{utsaga} är deklarativa meningar som kan klassificeras som 
antingen \emph{sanna} eller \emph{falska}.
Vi behöver inte alltid veta precis vilket, men det måste vara den ena eller den
andra -- aldrig båda.
Detta kallas inom logiken för \emph{lagen om det uteslutna tredje}\index{lagen 
om det uteslutna tredje}.
Om vi tittar på följande meningar:
\begin{enumerate}
  \item Denna text är skriven på svenska.
  \item Grön är en fin färg.
  \item Denna mening är falsk.
  \item Det finns oändligt många primtalstvillingar.
  \item \(x^2+1=0\)
\end{enumerate}

Den första meningen är en utsaga, och den är sann.
Den andra är ej en utsaga i logisk bemärkelse, det är en smaksak.
Den tredje meningen är ej en utsaga, den kan varken vara sann eller falsk
eftersom att det leder till motsägelsefulla slutsatser.
Den fjärde meningen är en utsaga, ingen vet dock om den är sann eller falsk.
Den femte och sista symbolföljden är en utsaga, men vi vet inte vad \(x\) är så
vi kan inte uttala oss om den skulle vara sann eller falsk.
Detta visar vikten av att tydligt specificera alla delar av en utsaga, så att
det är alldeles klart vad vi menar.
Den femte utsagan skulle behöva ändras till exempelvis \enquote{det finns ett
komplext tal \(x\) sådant att \(x^2+1=0\)} för att den skulle vara sann.
Om vi istället ändrat den till \enquote{det finns ett heltal \(x\) sådant att
\(x^2+1=0\)} skulle den vara falsk oavsett hur vi väljer \(x\) eftersom att
det inte finns ett sådant heltal.  En utsaga som alltid är falsk kallar vi för 
\emph{motsägelse}\index{motsägelse!utsaga} eller
\emph{kontradiktion}\index{kontradiktion|see {motsägelse}}.
En utsaga som alltid är sann kallar vi för \emph{tautologi}\index{tautologi}.

Två utsagor \(P\) och \(Q\) sägs vara logiskt 
ekvivalenta\index{ekvivalens!logisk} om \(P\) är sann
precis när \(Q\) är sann och följaktligen om \(P\) är falsk precis när \(Q\)
är falsk.
Vi skriver detta som \(P\equiv Q\).

\subsection{Kombinerade utsagor}

Vi vill också kunna forma nya utsagor från redan kända, detta genom att
kombinera och modifiera dem.
Om \(P\) är en utsaga, då säger vi att \emph{negationen}\index{negation!logisk} 
av \(P\), betecknad \(\lnot P\) eller \(\tnot P\), är falsk precis när \(P\) är 
sann och sann precis när \(P\) är falsk.
Vi kommer då fram till \emph{lagen om dubbelnegation}\index{lagen om 
dubbelnegation}.
Om vi funderar på vad som händer om vi tar \(\lnot(\lnot P)\) så kommer vi
fram till att \(P\equiv \lnot(\lnot P)\).
\begin{example}
  Ett exempel på negation, låt \(P\) vara utsagan \enquote{vi befinner oss i
  Sverige}.
  Då blir \(\lnot P\) \enquote{vi befinner oss inte i Sverige}.
  Vi kan också se att \(\lnot(\lnot P)\) blir \enquote{att vi inte befinner oss 
  inte i Sverige}, vilket är lite svårare att läsa men resulterar i att vi 
  måste befinna oss i Sverige.
\end{example}
\begin{example}
  Vi kan också titta på följande utsaga, \enquote{alla svenskar tycker om
  surströmming}.
  Negationen av den utsagan är \emph{inte} att ingen svensk tycker om
  surströmming, utan den är \enquote{inte alla svenskar tycker om 
  surströmming}.
  Det räcker då med att det finns en svensk som inte tycker om
  surströmming -- detta är en viktig skillnad att inte ta fel på!
\end{example}

Vi kan också kombinera utsagor genom \emph{konjunktioner}\index{konjunktion}.
Om \(P\) och \(Q\) är utsagor, då betecknar vi konjunktionen som \(P\tand Q\)
eller \(P\land Q\).
Konjunktionen är sann då både \(P\) och \(Q\) båda är sanna och falsk annars.
\begin{example}\label{ex:KonjunktionInternet}
  Låt \(P\) vara utsagan \enquote{jag bor i Sverige} och \(Q\) vara utsagan 
  \enquote{jag har en internetuppkoppling}.
  Då kan vi skapa den nya utsagan \(P\land Q\) som blir \enquote{jag bor 
    i Sverige \emph{och} jag har en internetuppkoppling}.
\end{example}

\begin{exercise}
  När är de olika utsagorna \(P\), \(Q\) och \(P\land Q\) i
  \cref{ex:KonjunktionInternet} sanna respektive falska?
\end{exercise}

Vi har också \emph{disjunktionen}\index{disjunktion} som betecknas \(P\tor Q\) 
eller \(P\lor Q\).
Disjunktionen är sann om antingen \(P\) eller \(Q\) eller båda är sanna, och är
således falsk endast när \(P\) och \(Q\) båda är falska.
\begin{example}
  Julfika kräver lussebullar \emph{eller} julfika kräver pepparkakor.
  Denna utsaga säger att om vi fikar lussebullar, då är det julfika.
  Den säger också att det är julfika om vi fikar pepparkakor.
  Den säger också att det är julfika om vi fikar både lussebullar och 
  pepparkakor samtidigt.
\end{example}

Konjunktionen och disjunktionen sammanfattas i en sanningstabell i
\cref{tbl:SanningKonjunktionDisjunktion}.

\begin{table}
  \caption{%
    Sanningstabell för konjunktionen och disjunktionen.
    S betecknar sant och F betecknar falskt.
  }
  \begin{tabular}{cccc}
    \(P\)  & \(Q\)   & \(P\land Q\)  & \(P\lor Q\) \\
    \toprule
    S      &  S      & S             & S \\
    S      &  F      & F             & S \\
    F      &  S      & F             & S \\
    F      &  F      & F             & F \\
    \bottomrule
  \end{tabular}\label{tbl:SanningKonjunktionDisjunktion}
\end{table}

\begin{exercise}\label{xrc:tartkalas}
  Vilken av följande logiska utsagor passar bäst för ett tårtkalas?
  På ett kalas
  \begin{enumerate}
    \item äts tårta \emph{och} dricks saft \emph{och} äts kakor
      \emph{och} dricks kaffe \emph{och} dricks te.
    \item äts tårta \emph{eller} dricks saft \emph{eller} äts kakor
      \emph{eller} dricks kaffe \emph{eller} dricks te.
  \end{enumerate}
\end{exercise}

\begin{exercise}
  Ingen av utsagorna i \cref{xrc:tartkalas} är egentligen riktigt bra 
  utsagor för att avgöra om något är ett tårtkalas.
  \begin{enumerate}
    \item Diskutera bristerna i de olika utsagorna.
    \item Utforma en egen bättre utsaga.
  \end{enumerate}
\end{exercise}

Hittills har vi kombinerat konjunktioner med konjunktioner och disjunktioner 
med disjunktioner, men det går naturligtvis även att kombinera konjunktioner 
med disjunktioner.
Vi visar här ett exempel på detta.

\begin{example}\label{ex:aldersgrans}
  I Sverige är åldersgränserna för film följande: under 7 år, under 11 år och 
  under 15 år~\cite{aldersgranser}.
  En film som inte får ses av barn under 11 år får ses av barn som uppfyller 
  följande utsaga: barnet är minst 11 år \emph{eller} barnet är minst 7 år 
  \emph{och} i vuxens sällskap.
  Notera att vi här kombinerat tre utsagor, först med disjunktion och sedan med 
  konjunktion.
\end{example}

\begin{exercise}
  Du kanske märkte av ett problem med utsagan i \cref{ex:aldersgrans}, om 
  inte kommer du att uppmärksamma det nu.
  Utsagan i exemplet har följande form: \(P\lor Q\land R\).
  Undersök om det spelar någon roll om vi tar \(P\lor Q\) eller \(Q\land R\) 
  först, det vill säga om \(P\lor (Q\land R)\equiv (P\lor Q)\land R\).
\end{exercise}

\begin{exercise}
  Testa att kombinera negationer, konjunktioner och disjunktioner, går det
  att forma några logiskt ekvivalenta utsagor?
\end{exercise}

\subsection{Implikationer}

Implikation är synonymt med ordet 
\emph{medför}\index{implikation}\index{medför|see {implikation}}.
Om \(P\) och \(Q\) är utsagor säger vi att \(P\) \emph{implicerar} \(Q\) eller
\emph{om \(P\), då \(Q\)}.
Vi ska undersöka när denna sammansatta utsaga bör vara sann och när den bör
vara falsk.
Låt oss formulera ett exempel.
\begin{example}
  Låt \(P\) vara utsagan \enquote{jag vinner pengar} och \(Q\) vara utsagan
  \enquote{jag köper nya böcker till skolan}.
  Utsagan \(P\implies Q\) blir då \enquote{\emph{om} jag vinner pengar,
  \emph{då} köper jag nya böcker till skolan}.
\end{example}
Implikationen är uppenbart falsk om jag vinner pengar men inte köper böcker till
skolan, men sann om jag köper böcker.
Annars, om jag inte vinner pengar, då har jag heller inte lovat att köpa
böcker till skolan.
Då måste utsagan vara sann i det fallet.
Men att jag inte vinner pengar hindrar mig ju inte att köpa böcker till
skolan ändå, följaktligen borde utsagan vara sann även i det fallet.
Det är detta resonemang som gjort att sanningsvärdena för implikationen är 
valda som de är.
Implikationens olika sanningsvärden sammanfattas i den vänstra delen av 
\cref{tbl:SanningImplikation}.

Vi kan naturligtvis vända på implikationen, om \(P\) och \(Q\) är utsagor och
\(P\implies Q\) då säger vi att dess \emph{invers}\index{invers!logisk} är 
\(Q\implies P\).
Inversen för en implikation är inte nödvändigtvis logiskt ekvivalent med
implikationen.
Ett exempel får illustrera.
\begin{example}\label{ex:kontrapositiv}
  Låt \(P\) vara utsagan \enquote{vi är i Stockholm} och \(Q\) vara utsagan 
  \enquote{vi är i Sverige}.
  Då blir \(P\implies Q\) utsagan \enquote{om vi är i Stockholm, då är vi i
  Sverige}.
  Dess invers \(Q\implies P\), \enquote{om vi är i Sverige, då är vi i
  Stockholm}, är däremot inte sann eftersom att vi skulle kunna vara i
  exempelvis Sundsvall, Göteborg eller Kiruna som också är städer i Sverige.

  Om vi däremot tittar på utsagan \(\lnot Q\implies \lnot P\), det vill säga
  \enquote{om inte vi är i Sverige, då är vi inte i Stockholm}, ser vi att 
  denna måste vara sann samtidigt som \(P\implies Q\).
\end{example}
Den senare utsagan i \cref{ex:kontrapositiv}, \(\lnot Q\implies \lnot P\), 
kallas för den \emph{kontrapositiva}\index{kontrapositiv!utsaga} utsagan av 
\(P\implies Q\).

Om \(P\implies Q\) och \(Q\implies P\) båda skulle vara sanna, vilket ibland är 
fallet, då skriver vi detta som \(P\iff Q\).
Utsagan \(P\iff Q\) kallas för 
\emph{dubbelimplikation}\index{dubbelimplikation|see {logisk ekvivalens}} eller
\emph{ekvivalens}\index{ekvivalens!logisk} och är sann då \(P\) och \(Q\) båda 
är sanna och då de båda är falska.
Den utläses som \emph{\(P\) om och endast om \(Q\)}\index{om och endast om|see 
{logisk ekvivalens}}.
\enquote{Om och endast om} förkortas ibland som \enquote{omm}.
Vi kan se att \(\iff\) får samma betydelse som \(\equiv\).

\begin{exercise}
  Undersök några olika logiska utsagor, likt den kontrapositiva utsagan, och se 
  om du kan finns några intressanta ekvivalenser.
\end{exercise}

Vi ska nu avsluta med en viktig logisk ekvivalens till implikationen.
Denna ligger till grund för motsägelsebevis.
Utsagan \((P\land\lnot Q)\implies C\), där \(C\) är en motsägelse och
därmed alltid är falsk, är logiskt ekvivalent med \(P\implies Q\).
Detta ses tydligast i en sanningstabell, och den är given i
\cref{tbl:SanningImplikation} tillsammans med implikationen och dess
kontrapositiva utsaga.
Implikationen \(P\implies Q\) är bara falsk när \(P\) är sann och \(Q\) är
falsk.
Konjunktionen \(P\land\lnot Q\) är sann endast när \(P\) är sann och \(Q\) är
falsk.
Om \(C\) alltid är falsk, då kommer \(P\land\lnot Q\implies C\) att vara falsk
endast när \(P\land\lnot Q\) är sann.
Men det är ju precis när \(P\) är sann och \(Q\) är falsk, det vill säga när
\(P\implies Q\) är falsk.
Följaktligen måste de vara logiskt ekvivalenta.

\begin{table}
  \caption{%
    Sanningstabell för implikationen och dess logiskt ekvivalenta former.
    S betecknar sant och F betecknar falskt.
  }
  \begin{tabular}{ccccccc}
    \(P\) &
    \(Q\) &
    \(P\implies Q\) &
    \(\lnot(P\land \lnot Q)\) &
    \(\lnot Q\implies \lnot P\) &
    \(C\) & \((P\land\lnot Q)\implies C\) \\
    \toprule
    S & S & S & S & S & F & S \\
    S & F & F & F & F & F & F \\
    F & S & S & S & S & F & S \\
    F & F & S & S & S & F & S \\
    \bottomrule
  \end{tabular}\label{tbl:SanningImplikation}
\end{table}


%%%%%%%%%%%%%%%%%%%%%%%%%%%%%%%%%%
% AXIOM, SATSER OCH BEVIS
%%%%%%%%%%%%%%%%%%%%%%%%%%%%%%%%%%
\section{Axiom, satser och bevis}

För att det ska kunna gå att härleda någonting måste det finnas
några grundläggande utsagor som en grund att bygga på.
Dessa utsagor kallar vi för \emph{axiom}\index{axiom}, och det är från dessa
alla matematiska härledningar utgår.
Vi har också definitioner som indirekt kan specificera axiom.

Vi har ovan redan sett några axiom, nämligen de logiska axiomen.
Det vill säga sanningstabellerna vi tittat på ovan, de fungerar som axiom för 
logiken.
Vi var dock inte tydliga med att dessa var axiom eftersom att vi inte visste 
vad ett axiom var då.
Vi ska i kommande kapitel ta upp de matematiska axiomen.
Det finns axiom som används i nästan all matematik, dessa är axiomen för
mängdläran (\cref{ch:Mangder}), och det finns olika axiomuppsättningar 
inom specifika områden inom matematiken.
Exempelvis ska vi se en axiomuppsättning för geometri 
i \cref{ch:Geometri}, denna gäller för klassisk geometri, det vill säga 
geometri i plana ytor.
I sfärisk geometri, geometri på ytan av en rund boll, som vi inte tar upp 
i denna volym, byts några av axiomen ut.

Grunden må vara viktig att stå på, men möjligheten att ta oss vidare till nya 
resultat -- satser -- är också av yttersta vikt.
Faktum är ju att vi tidigare behövde logiken för att kunna resonera och dra
slutsatser, för att kunna bevisa nya resultat.
Nya resultat, som är implikationer eller dubbelimplikationer, sammanfattas i
något som kallas för \emph{satser}\index{sats}.
En sats ges oftast på formen \enquote{om dessa villkor \(P\) är uppfyllda, då 
gäller även \(Q\)}, där \(P\) och \(Q\) är utsagor.
Men en sats kan inte bara presenteras utan vidare, den kräver alltid ett
\emph{bevis}\index{bevis}.
Ett bevis är en logisk härledning som utgår från axiomen och andra tidigare
bevisade satser för att visa att om \(P\) är sann då måste även \(Q\) vara sann
precis då.
Det vill säga visa att implikationen är sann.

Satsen är huvudbegreppet, men vi har även andra typer av satser.
Vi har \emph{lemman}\index{lemma}, som är hjälpsatser\index{hjälpsats}.
De är också satser, men av mindre betydelse.
Dessa behöver vi för att visa ett mindre resultat för att beviset för en annan
sats inte ska bli onödigt långt.
Vi har även \emph{korollarier}\index{korollarium}, som är
följdsatser\index{följdsats}.
Detta är satser som följer mer eller mindre direkt från en annan sats och har
därför ett mycket kort bevis.

Vi ska nu titta på några vanliga bevismetoder.
När ett bevis genomförs och presenteras brukar detta avslutas med 
\textsc{Q.E.D.}, som är en förkortning för latinets \emph{Quod Erat
Demonstrandum} och betyder \enquote{vilket skulle visas}.
Detta är ett arv från tiden då latin var det vetenskapliga språket och mer
eller mindre all vetenskaplig kommunikation skedde på latin.
Det skulle kunna jämföras med engelskans position idag.

Vi kommer nedan att beskriva den bakomliggande idén för respektive bevismetod.
Därefter kommer vi att se dem tillämpas i resterande kapitel i boken.

\subsection{Motexempelbevis}

Vi börjar med den enklaste bevismetoden.
Om någon skulle påstå att \enquote{alla svenskar tycker om surströmming}, då 
räcker det med att vi hittar en svensk som inte tycker om surströmming för att
motbevisa påståendet.
Det vill säga, vi hittar ett motexempel\index{motexempelbevis}.
Kom ihåg från tidigare att negationen av utsagan \enquote{alla svenskar tycker 
om surströmming} är \enquote{det finns åtminstone en svensk som inte tycker om
surströmming} och att det är denna utsaga som vi bevisar genom att finna en
sådan svensk.

\subsection{Direkta bevis}

Vi låter \(P\) och \(Q\) vara utsagor.
För att \emph{hypotesen}\index{hypotes} \(P\) ska implicera 
\emph{konklusionen}\index{konklusion} \(Q\) måste \(P\) vara sann
precis när \(Q\) är sann.
Vi åstadkommer detta genom konstruktionen av en kedja av implikationer
\[
  P\implies R_1, R_1\implies R_2, \ldots, R_n\implies Q.
\]
Enligt \emph{lagen om syllogism}\index{lagen om syllogism} måste då \(P\implies 
Q\).
Karakteristiskt för denna bevismetod är att det bara är att \enquote{jobba på} 
för att komma fram till konklusionen.

\begin{exercise}
  Övertyga dig själv med hjälp av sanningstabeller
  \begin{enumerate}
    \item att \(P\implies R_1\) och \(R_1\implies R_2\) och \(R_2\implies Q\) 
      är ekvivalent med \(P\implies R_1\implies R_2\implies Q\), därefter
    \item att \(P\implies Q\) är ekvivalent med \(P\implies R_1\implies 
      R_2\implies \cdots\implies Q\).
  \end{enumerate}
\end{exercise}

\subsection{Kontrapositiva bevis}

Låt \(P\) och \(Q\) vara utsagor.
Eftersom att vi tidigare, i \cref{tbl:SanningImplikation}, sett att den
kontrapositiva\index{kontrapositiv!bevis} implikationen \(\lnot Q\implies\lnot 
P\) är logiskt ekvivalent
med \(P\implies Q\) kan vi likväl bevisa den kontrapositiva implikationen som
\(P\implies Q\).
Vi vill kunna göra detta för att ibland kan det vara lättare att visa än \(P\) 
medför \(Q\).

\subsection{Motsägelsebevis}

Motsägelsebeviset\index{motsägelse!bevis} och det direkta beviset är kanske de 
bevismetoder som används flitigast i denna bok.
Motsägelsebeviset är effektivt och kan ofta vara enklare att använda än att
konstruera ett direkt bevis.
Metoden använder en logisk ekivalens, precis som föregående metod, nämligen att
\((P\land\lnot Q)\implies C\) är logiskt ekvivalent med \(P\implies Q\) när
\(C\) är en motsägelse.
Den säger att vi ska anta vår hypotes \(P\) och även anta motsatsen
\(\lnot Q\) till vår önskade konklusion \(Q\).
Om dessa antaganden tillsammans leder till en utsaga som alltid är falsk, det
vill säga en motsägelse \(C\), då har vi visat att \(P\) implicerar \(Q\)
eftersom att detta är logiskt ekvivalent.


\section{Fördjupande läsning}

För vidare läsning om logik, se exemeplvis bilaga A i \emph{Introduction to 
real analysis} av \citet{Bartle2000itr} eller, för ytterligare fördjupning, del 
A i \emph{Foundations of logic and mathematics} av 
\citet{nievergelt2002foundations}.

