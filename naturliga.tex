\chapter{De naturliga talen}\label{DeNaturligaTalen}

\lettrine{D}{e naturliga talen} är de tal som vi använder för att ordna
och räkna saker.
Talen \(1,2,3,\ldots\) är naturliga tal och det är dessa tal som människan
använt längst i historien.
Talet noll kom väldigt sent i historien, så sent som på 800-talet i Indien,
medan de andra talen använts sedan årtusenden tillbaka i tiden.
Vi ska ändå ta med talet \(0\) bland våra naturliga tal.
Mängden av naturliga tal betecknas \(\N\), det vill säga
\(\N=\{0,1,2,3,\ldots\}\).
Då dessa tal funnits länge i människans historia har de också betraktats som
självklara, så självklara att den tyske matematikern Leopold Kronecker 
(1823--1891) sade \enquote{Den käre Gud har skapat de hela talen, allt annat är 
  människans verk}~\cite[egen översättning]{Kline1990mtf3}.

Under 1800-talet blev dock matematikerna uppmärksamma på att det behövdes en
stadigare grund att bygga matematiken på.
Det gick inte längre att anta talen som självklara.
Hittills hade de naturliga talen (detta kapitel), heltalen 
(\cref{ch:Heltalen}) och de reella talen (\cref{ch:Reella}) ansetts 
vara självklara.
Men nu började självklarheten att ifrågasättas.

I detta kompendium kommer grunden att läggas först och sedan fortsätter vi
vår väg uppåt.
Det vill säga, vi börjar med de naturliga talen och går sedan vidare till de
hela talen, de rationella talen och slutligen de reella talen.
Att döma av det historiska förloppet grundades matematiken egentligen i omvänd
ordning.
Det vill säga, de reella talen grundades först.
Sedan se rationella talen, de hela talen och sist de naturliga talen.
%De reella talen var först att grundas, de grundades på de rationella talen.
%De rationella talen grundades därefter på heltalen.
%Heltalen grundades därefter på de naturliga talen.
%Slutligen grundades de naturliga talen genom de axiom som tas upp i detta
%kapitel.
%Analogt kan sägas att taket på huset byggdes innan grunden var lagd och
%väggarna resta.
Vi ska nu i kommande avsnitt titta närmare på de axiom som ligger till grund
för de naturliga talen.


%%%%%%%%%%%%%%%%%%%%%%%%%%%%%%%%%%%
% PEANOS AXIOM
%%%%%%%%%%%%%%%%%%%%%%%%%%%%%%%%%%%
\section{Peanos axiom för de naturliga talen}
Under 1800-talets slut och början av 1900-talet grundades de delar
av matematiken som redan använts sedan årtusenden tillbaka.
De naturliga talen fick sin axiomatiska grund när Richard Dedekind (1831--1916)
år 1888 publicerade ett antal axiom för de naturliga talen.
Året efter publicerade dock Giuseppe Peano (1858--1932) en förbättring av dessa
axiom och det är Peanos förbättrade axiom som godtagits och används idag,
om än i lite annorlunda formulering.

%Vi ska även se i \cref{sec:vonNeumannNaturliga} att Peanos axiom
%faktiskt kan härledas ur mängdteorin och att mängdteorin därför kan användas
%som fundamental teori för matematiken.
%Dedekind och Peanos axiom kan ändå användas som grund för de naturliga talen,
%men istället för att vara axiom blir de satser som kan härledas i mängdteorin.

Vi ska nu titta på Peanos axiom för de naturliga talen.
Börja med att släppa taget om allt du tror dig känna till om matematiken.
När du fortsätter efter denna mening ska din \enquote{matematikvärld} vara helt 
tom.
Därefter kan du fylla den med axiomen alltefter de presenteras i texten.
Vi börjar nu med det första axiomet.
\begin{axiom}\label{ax:NaturligaAxiomNoll}
  \(0\) är ett naturligt tal. 
\end{axiom}
Det första axiomet, \cref{ax:NaturligaAxiomNoll}, säger helt enkelt att det
finns åtminstone ett naturligt tal.
Det säger ingenting mer om \(0\) än att det är ett naturligt tal och
vi vet inte ännu vilka egenskaper som nollan besitter.
Detta är allt som nu finns i vår matematikvärld --- \enquote{noll är ett 
  naturligt tal}.
Vi går vidare till nästa axiom.

\begin{axiom}\label{ax:NaturligaAxiomEfterfoljare}
  För alla naturliga tal \(a\) existerar en efterföljare betecknad \(S(a)\)
  som är ett naturligt tal.
\end{axiom}
För ett naturligt tal \(n\), låter vi dess efterföljare betecknas med \(S(n)\).
På samma sätt låter vi \(S(S(n))\) beteckna efterföljaren till efterföljaren
till \(n\), och så vidare.
Notera att vi vet ännu inte vad som menas med en efterföljare.
Det enda vi vet är att alla naturliga tal har en efterföljare som är ett
naturligt tal.
Vi kan därmed fylla upp vår matematikvärld med en lång rad efterföljare och
efterföljare till efterföljare och så vidare.
\begin{axiom}\label{ax:NaturligaAxiomEjCirkular}
  För alla naturliga tal \(n\) gäller att \(0\) inte är dess efterföljare.
\end{axiom}
Vi kommer att återkomma till de två senaste axiomen.
Men låt oss först gå vidare med vad vi menar med likhet och att två naturliga
tal är lika.
Vi betecknar likhet med tecknet \(=\).
\begin{axiom}[Reflexivitet]\label{ax:NaturligaAxiomReflexivitet}
  För alla naturliga tal \(n\) gäller att \(n\) är lika med sig själv.
  Detta betecknas \(n=n\).
\end{axiom}
Vi kan nu också konstatera i vår matematikvärld att \(0=0\), och vi vet att
\(S(0)=S(0), S(S(0)) = S(S(0)), \ldots, S(S(\ldots S(0))) = S(S(\ldots
S(0)))\).
\begin{axiom}[Symmetri]\label{ax:NaturligaAxiomSymmetri}
  För alla naturliga tal \(a\) och \(b\) gäller att om \(a\) är lika med
  \(b\) då är även \(b\) lika med \(a\).
  Det vill säga, om \(a=b\) då är \(b=a\).
\end{axiom}
\begin{axiom}[Transitivitet]\label{ax:NaturligaAxiomTransitivitet}
  För alla naturliga tal \(a,b\) och \(c\) gäller att
  om \(a=b\) och \(b=c\) då är \(a=c\).
\end{axiom}
\begin{axiom}[Slutenhet under likhet]\label{ax:NaturligaAxiomSlutenhet}
  För alla naturliga tal \(a\) gäller att om \(a=b\) för något \(b\) då måste
  \(b\) också vara ett naturligt tal.
\end{axiom}
Axiomen~\ref{ax:NaturligaAxiomReflexivitet},~\ref{ax:NaturligaAxiomSymmetri},~\ref{ax:NaturligaAxiomTransitivitet} 
och~\ref{ax:NaturligaAxiomSlutenhet}
behandlar begreppet likhet (\(=\)).
Axiom~\ref{ax:NaturligaAxiomReflexivitet} säger att ett naturligt tal måste
vara lika med sig självt.
Denna egenskap kallas reflexivitet\index{reflexivitet}.
Axiom~\ref{ax:NaturligaAxiomSymmetri} säger att om ett naturligt tal är lika
med ett annat, då måste även omvändningen gälla.
Denna egenskap kallas symmetri\index{symmetri}.
Axiom~\ref{ax:NaturligaAxiomTransitivitet} säger att om vi får en kedja med
likheter, då måste ändarna av kedjorna vara lika.
Exempelvis, om \(a=b\) och \(b=c\) får vi att \(a=b=c\) och \(a=c\) måste då
gälla.
Denna egenskap kallas transitivitet\index{transitivitet}.
Axiom~\ref{ax:NaturligaAxiomSlutenhet} säger att om ett naturligt tal är lika
med någonting, då måste detta någonting också vara ett naturligt tal.
Denna egenskap kallas slutenhet\index{slutenhet}, hur vi än använder likhet kan
vi inte komma utanför de naturliga talen.
Vi vet nu hur begreppet likhet och \(=\) ska fungera och vad det betyder.

Vi ska nu introducera ytterligare ett axiom som vi vill kombinera med
tidigare axiom.
\begin{axiom}\label{ax:NaturligaAxiomInjektion}
  För alla naturliga tal \(a\) och \(b\) gäller att om deras efterföljare är
  lika måste även \(a\) och \(b\) vara lika.
  Det vill säga, om \(S(a)=S(b)\) då är \(a=b\).
\end{axiom}
Vi ska nu gå tillbaka till \cref{ax:NaturligaAxiomEfterfoljare} och
\cref{ax:NaturligaAxiomEjCirkular}.
Axiom~\ref{ax:NaturligaAxiomInjektion} säger att två olika tal kan inte ha 
samma efterföljare.
Detta betyder att vi inte kan få exempelvis grenstruktur eller ''öglor''.
Utan strukturen som måste uppstå är en linje där varje naturligt tal är en 
efterföljare till ett unikt annat naturligt tal --- med undantag för noll 
(\(0\)) som enligt \cref{ax:NaturligaAxiomEjCirkular} inte är efterföljare 
till något naturligt tal.
Axiom~\ref{ax:NaturligaAxiomEfterfoljare} säger att ett naturligt tals 
efterföljare alltid är ett naturligt tal och att en sådan alltid existerar.
Dessa två axiom säger tillsammans med \cref{ax:NaturligaAxiomInjektion} att
det finns oändligt många naturliga tal.
Om vi har ett naturligt tal kan vi alltid ta dess efterföljare enligt 
\cref{ax:NaturligaAxiomEfterfoljare}, men oavsett hur många efterföljare 
vi tar kommer vi enligt \cref{ax:NaturligaAxiomEjCirkular} aldrig tillbaka 
dit vi startade vid \(0\).
Vi vet att inget naturligt tal kan ha \(0\) som efterföljare, men \(S(0)\) då?
Om vi låter \(S(S(0))\) ha \(S(0)\) som efterföljare, då får vi en ögla trots
att det inte har \(0\) som efterföljare.
Därför behöver vi \cref{ax:NaturligaAxiomInjektion} som säger att då måste
\(S(S(0))\) och \(0\) vara samma naturliga tal --- vilket inte är sant och
följaktligen kan vi inte få några öglor.

Vi tittar nu på det sista axiomet.
\begin{axiom}[Induktionsaxiomet]\label{ax:NaturligaAxiomInduktion}
  Låt \(M\) vara en samling av objekt sådan att \(0\) tillhör \(M\) och har
  egenskapen att det för alla naturliga tal \(n\) gäller att om \(n\) tillhör
  samlingen \(M\) då tillhör även efterföljaren \(S(n)\) samlingen \(M\).
  Då innehåller \(M\) alla naturliga tal.
\end{axiom}
Det sista axiomet, \cref{ax:NaturligaAxiomInduktion}, beskriver
induktionsprincipen, de naturliga talen i sig och även mängden av alla
naturliga tal.
Det säger att om \(0\) tillhör en samling och efterföljaren till varje
naturligt tal i samlingen finns med, då innehåller mängden alla
naturliga tal.
Noll (\(0\)) är ett naturligt tal, då finns efterföljaren \(S(0)\) också med.
Eftersom att efterföljaren \(S(0)\) till \(0\) är ett naturligt tal, då måste
även \(S(S(0))\) vara med i denna samling.
Då säger vi att samlingen måste innehålla alla naturliga tal.
Det följer också från detta axiom att alla naturliga tal är på formen
\(S(S(\ldots S(0)))\).
Det är detta axiom som ligger till grund för bevismetoden induktion, därav
axiomets namn.
\begin{exercise}
  Genom diskussion jämför induktionsaxiomet,
  \cref{ax:NaturligaAxiomInduktion}, med hur induktionsbevis från
  \cref{ch:Mangder} genomförs.
\end{exercise}

Vi ska nu införa några välbekanta symboler.
\begin{definition}\label{def:NaturligaBeteckningar}
  Låt följande symboler beteckna de olika efterföljarna.
  \begin{equation*}
    1 = S(0),\quad
    2 = S(1),\quad
    3 = S(2),\quad
    \ldots
  \end{equation*}
  Låt dessutom \(\N=\{0,1,2,3,\ldots\}\) beteckna mängden av alla naturliga
  tal.
\end{definition}
\begin{remark}
  Märk väl att \(1,2\) och \(3\) enbart utgör symboler för
  \[
  S(0),S(S(0))\text{ respektive }S(S(S(0))).
  \]
  Vi vet inte hur dessa förhåller sig till varandra genom addition och
  multiplikation eftersom vi inte vet vad addition och multiplikation är
  ännu.
  Vi vet inte ens om dessa objekt som symbolerna representerar går att räkna.
  Ännu utgör dessa bara en oordnad ansamling av symboler.
\end{remark}

% TODO fixa definition av S^n(0) = S(S(...S(0)))
Vi ska nu göra en definition som kommer att förenkla vår notation avsevärt.
Denna definition använder faktiskt induktionsprincipen från induktionsaxiomet.
\begin{definition}\label{def:Efterfoljarpotenser}
  Låt \(S^0(0) = 0\).
  Om \(S^n(0)\) är definierad för något naturligt tal \(n\), då låter vi
  \(S^{S(n)}(0) = S(S^n(0))\).
\end{definition}
\begin{example}
  Vi ska titta närmare på det naturliga talet \(3=S^3(0)\).
  Vi har att \(0=S^0(0)\) och \(1=S(0)\), men då måste \(S(0)=S(S^0(0))\).
  Enligt \cref{def:Efterfoljarpotenser} är detta samma sak som
  \(S^{S(0)}(0)=S^1(0)\).
  Vi har också att \(2=S(S(0))\).
  På samma sätt som tidigare får vi att \(S(S(0))=S(S^1(0))\) och vi kan nu
  fortsätta med att konstatera att detta är lika med \(S^{S(1)}(0)=S^2(0)\).
  Vi kan nu använda detta resultat för att se att \(3=S(S(S(0)))\) faktiskt
  är \(S^3(0)\).
\end{example}


%%%%%%%%%%%%%%%%%%%%%%%%%%%%%%%%%%%%%%
%% VON NEUMANNS KONSTRUKTION
%%%%%%%%%%%%%%%%%%%%%%%%%%%%%%%%%%%%%%
%\section{Von Neumanns konstruktion av de naturliga talen}
%\label{sec:vonNeumannNaturliga}
%\dots


%%%%%%%%%%%%%%%%%%%%%%%%%%%%%%%%%%%%%
% ARITMETIK
%%%%%%%%%%%%%%%%%%%%%%%%%%%%%%%%%%%%%
\section{Aritmetik}%
\index{aritmetik}
Ordet aritmetik kommer från grekiskans \emph{\ibygr{a)riqmo)s}},
som betyder tal, och \emph{\ibygr{a)riqmhtikh)}}, som betyder konsten att
räkna~\cite{OED2013arithmetic}.
Aritmetiken kan beskrivas som läran om att kombinera tal.
De delar av aritmetiken vi ska behandla i detta avsnitt är operationerna
addition (\(+\)) och multiplikation (\(\cdot\)).
Det vill säga, vi ska i detta avsnitt bestämma hur man räknar med de naturliga
talen.

Innan vi går vidare till att titta på addition och multiplikation behöver vi 
dock veta vad en binär operation faktiskt är.
\begin{definition}\index{binär operation}
  En \emph{binär operation}\index{binär operation} \(\diamond\) på en mängd
  \(M\) är en funktion \(\diamond\colon M\times M\to M\) som tar två
  element \(x\) och \(y\) i \(M\) och parar dessa med ett element
  \(\diamond(x,y)\) i \(M\).
  Vanligtvis betecknas \(\diamond(x,y)\) med \(x\diamond y\).
  Det vill säga, \((x,y)\mapsto x\diamond y\).
\end{definition}


%%%%%%%%%%%%%%%%%%%%%%%%%%%%%%%%%%%%%
% ADDITION
%%%%%%%%%%%%%%%%%%%%%%%%%%%%%%%%%%%%%
\subsection{Addition}
Den första av de aritmetiska operationerna vi ska ta upp är addition.
Den definition vi använder oss av här är samma definition som gavs av Peano år 
1889.
Peanos definition av addition bygger även den på induktionsprincipen och kan
därför till en början kanske upplevas lite underlig och svårförståelig, men vi
ska diskutera den efteråt.
\begin{definition}[Summa]\label{def:NaturligaSumma}\index{summa}\index{term}\index{\(+\)}
  För varje par av naturliga tal \(a\) och \(b\) definieras en \emph{summa}
  \(a+b\) som är ett naturligt tal.
  Delarna \(a\) och \(b\) av en summa kallas för summans \emph{termer}.
  Vi definierar först
  \begin{equation}
    \label{eq:AdditionNoll}
    a+0 = a.
  \end{equation}
  Om summan \(a+b\) är definierad låter vi
  \begin{equation}
    \label{eq:AdditionRekursion}
    a+S(b) = S(a+b).
  \end{equation}
\end{definition}
Den första delen av definitionen är tämligen enkel.
Allt \cref{eq:AdditionNoll} säger är att om vi adderar noll från höger till
ett tal så får vi talet självt.
Det vill säga, det händer ingenting vid addition med noll från höger.
Detta är dock väldigt viktigt, och vi kommer att se varför alldeles strax. 

Den andra delen kan upplevas lite svårare.
Det \cref{eq:AdditionRekursion} säger är att ett tal adderat med efterföljaren
till ett annat är samma sak som efterföljaren till de båda talens summa.
Men hur hjälper det oss?
Det visas lättast med ett exempel.
\begin{example}
  Vi vill finna summan för talen \(2\) och \(3\).
  Alla tal kan skrivas som en kedja av efterföljare till noll, vi vet från
  \cref{def:NaturligaBeteckningar} att \(2=S(S(0))\) och \(3=S(S(S(0)))\).
  Om vi skriver summan \(2+3\) på formen från
  \cref{eq:AdditionRekursion} har vi \(2+S(S(S(0)))=S(2+S(S(0)))\).
  Men då fick vi ett nytt uttryck \(2+S(S(0))\) som är på samma form, summan
  av ett tal och efterföljaren till ett tal.
  Om vi använder \cref{eq:AdditionRekursion} igen får vi
  \(S(2+S(S(0)))=S(S(2+S(0)))\).
  Nu har vi återigen ett uttryck på samma form.
  Upprepning ger oss \(S(S(2+S(0)))=S(S(S(2+0)))\).
  Nu fick vi dock inte en summa av ett tal och en efterföljare, utan
  vi fick summan av ett tal och noll.
  Men vi vet ju från \cref{eq:AdditionNoll} att \(2+0=2\) och då får vi
  \(S(S(S(2)))\).
  Det är just detta som gör \cref{eq:AdditionNoll} så viktig, förr eller
  senare kommer vi fram till en summa där ena termen är noll och då måste vi
  veta vad det är.
  Utöver detta vi vet också att \(2=S(S(0))\), om vi sätter in detta får vi
  \(S(S(S(S(S(0)))))\) som vi enligt definition betecknar med \(5\).
  Följaktligen är \(2+3=5\).
\end{example}

Denna typ av återupprepande användning av sig själv kallas för
\emph{rekursion}\index{rekursion}.

\begin{exercise}
  Visa att om \(a\) är ett naturligt tal, då är \(a+1=S(a)\).
\end{exercise}
\begin{exercise}
  Visa att om \(a\) och \(b\) är naturliga tal, då är \(a+b=S^b(a)\).
\end{exercise}

Notera att \(a+0\) per definition är lika med \(a\), detta säger tyvärr
ingenting om \(0+a\).
\begin{exercise}
  Är \(0+a\) också lika med \(a\)?
  Bevisa ditt påstående.
\end{exercise}

Om vi studerar summan ser vi att \(+\) är en binär operation på mängden av
naturliga tal.
Vi kallar denna operation för \emph{addition}.
\begin{definition}[Addition]\label{def:NaturligaAddition}\index{addition}\index{naturliga 
    tal!addition}
  Additionsoperatorn \(+\) är en funktion \(+\colon\N\times\N\to\N\) sådan att
  varje par av naturliga tal \((a,b)\) avbildas på summan \(a+b\), som är
  ett naturligt tal som vi finner genom \cref{def:NaturligaSumma}.
\end{definition}



%%%%%%%%%%%%%%%%%%%%%%%%%%%%%%%%%%%%%
% IDENTITETSELEMENT
%%%%%%%%%%%%%%%%%%%%%%%%%%%%%%%%%%%%%
\subsection{Identitetselementet}
När vi nu sett nollans speciella betydelse är det dags att ge dess viktiga
egenskap ett namn.
Detta gör vi i följande definition.
\begin{definition}\index{identitetselement}%\index{enhet}
  Givet en mängd \(M\) med en definierad binär operation \(\diamond\colon 
  M\times M\to M\).
  Ett element \(e\) kallas \emph{identitetselement} om det för alla element
  \(x\) i mängden uppfyller att \(x\diamond e=e\diamond x=x\).
\end{definition}
\begin{example}
  Det naturliga talet \(0\) är det additiva identitetselementet för de
  naturliga talen.
\end{example}
Man kan nu undra varför vi vill ha en sådan definition enbart för nollan?
Anledningen är att det finns andra tal bland de naturliga talen som beter
precis som nollan, fast för en annan operation än addition.
Vi kommer att stöta på ett identitetselement till i nästa avsnitt.


%%%%%%%%%%%%%%%%%%%%%%%%%%%%%%%%%%%%%
% MULTIPLIKATION
%%%%%%%%%%%%%%%%%%%%%%%%%%%%%%%%%%%%%
\subsection{Multiplikation}
Multiplikation är den andra aritmetiska operationen vi ska titta på i detta
kapitel.
Även definitionen av multiplikation är den Peano gav år 1889.
Dessutom bygger även den på rekursion, precis som definitionen för addition.
\begin{definition}[Produkt]\label{def:NaturligaProdukt}\index{produkt}\index{faktor}\index{\(\cdot\)}
  För varje par av naturliga tal \(a\) och \(b\) definierar vi en
  \emph{produkt} \(a\cdot b\) som är ett naturligt tal.
  %Det vill säga \((a,b)\in\N\times\N\) och \(a\cdot b\in\N\), och alltså
  %\(\cdot\colon\N\times\N\to\N\) där \((a,b)\mapsto a\cdot b\).
  Delarna \(a\) och \(b\) av en produkt kallas för produktens \emph{faktorer}.
  Vi definierar först
  \begin{equation}
    a\cdot 0 = 0.
  \end{equation}
  Om produkten \(a\cdot b\) är definierad låter vi
  \begin{equation}
    a\cdot S(b) = a+(a\cdot b).
  \end{equation}
  Produkten \(a\cdot b\) skrivs vanligen som \(ab\).
\end{definition}

Likt summan ger även produkten en binär operation.
\begin{definition}[Multiplikation]\label{def:NaturligaMultiplikation}\index{multiplikation}\index{naturliga 
    tal!multiplikation}
  Multiplikationsoperatorn \(\cdot\) är en funktion
  \(\cdot\colon\N\times\N\to\N\) sådan att varje par av naturliga tal
  \((a,b)\) avbildas på produkten \(a\cdot b\), som är ett naturligt tal som
  vi finner genom \cref{def:NaturligaProdukt}.
  Denna operation kallar vi för \emph{multiplikation}.
\end{definition}

\begin{exercise}
  Vilket element är identitetselementet för multiplikation av de naturliga
  talen?
  Visa att så är fallet.
\end{exercise}
\begin{exercise}
  Visa att om \(a\) och \(b\) är naturliga tal, då är \[a\cdot b =
  \underbrace{a+a+\cdots+a}_b.\]
\end{exercise}
\begin{exercise}
  Visa att \(0\cdot a = 0\).
  Notera att \(a\cdot 0\) per definition är lika med \(0\), \(0\cdot a = 0\)
  fordrar dock ett bevis.
\end{exercise}


%%%%%%%%%%%%%%%%%%%%%%%%%%%%%%%%%%%%%
% LIKHET OCH OLIKHET
%%%%%%%%%%%%%%%%%%%%%%%%%%%%%%%%%%%%%
\section{Likhet och olikhet}
Det är nu dags att introducera ett sätt att jämföra tal som ej är
lika.
Vi har redan sett likhet, som vi betecknade med \(=\) och utläste \emph{är lika
med}.
Likheter är väldigt intressanta, men det finns många saker som inte är lika.
Exempelvis har vi de naturliga talen, de är oändligt många och inget av dem är
lika med något annat.
Därför definierar vi här en annan relation på de naturliga talen.

\begin{definition}[Olikhet]\index{olikhet}\index{naturliga 
    tal!olikhet}\index{<}\index{\(\leq\)}\index{>}\index{\(\geq\)}
  Låt \(a\) och \(b\) vara naturliga tal.
  Då säger vi att \(a\) är \emph{mindre än eller lika med} \(b\) om det finns
  ett naturligt tal \(n\) sådant att \(a+n=b\), vi skriver detta som
  \(a\leq b\).
  Vi kan också säga att \(b\) är \emph{större än eller lika med} \(a\) och
  beteckna detta genom \(b\geq a\).
  Om vi ej tillåter \(n\) att vara noll, då skriver vi \(a<b\)
  respektive \(b>a\).
  Vi utläser dessa som \(a\) är \emph{strikt mindre än} \(b\)
  respektive \(b\) är \emph{strikt större än} \(a\).
\end{definition}

\begin{example}
  Vi kan nu säga att \(0<1<2<3\) och så vidare.
  Det vill säga, vi har nu infört en form av ordning av de naturliga talen.
\end{example}

\begin{example}
  Låt \(x\) vara ett naturligt tal sådant att \(0<x\) och \(x<5\), det vill
  säga \(0<x<5\).
  Vi menar då att \(x\) kan vara något av talen \(1,2,3\) eller \(4\).
\end{example}

\begin{exercise}
  Är \(<\), \(\leq\), \(>\) och \(\geq\) relationer?
\end{exercise}


%%%%%%%%%%%%%%%%%%%%%%%%%%%%%%%%%%%%%
% EGENSKAPER FÖR ADDITION
%%%%%%%%%%%%%%%%%%%%%%%%%%%%%%%%%%%%%
\section{Additionens algebraiska egenskaper}

% XXX Ange referens för al-jabr
Ordet algebra kommer från arabiskans \emph{al-jabr} genom Muhammad
ibn M\={u}s\={a} al-Khw\={a}rizm\={\i}s (ca. 780-ca. 850) bok \emph{Al-Kit\={a}b
al-mukhta\d{s}ar f\={\i} h\={\i}s\={a}b al-\u{g}abr wa'l-muq\={a}bala}, på
svenska \emph{Den sammanfattande boken om beräkning genom komplettering och
  balansering}\footnote{%
  Egen översättning från \enquote{The Compendious Book on
  Calculation by Completion and Balancing}.
}.
Ordet \emph{al-jabr} betyder ordagrant \emph{återställande}.
Algebra kan beskrivas som matematikens studium av operationer och regler.
Vi ska nu titta närmare på hur den aritmetiska operationen addition beter sig.

Innan vi tittar på additionens algebraiska egenskaper behöver vi några
hjälpsatser.
Vi börjar med en mycket enkel hjälpsats som säger att om man adderar talet ett
till ett tal får man dess efterföljare.

\begin{lemma}\label{NaturligaAdderaEtt}
  Om \(a\) är ett naturligt tal, då är \(a+1\) dess efterföljare.
\end{lemma}
\begin{proof}
  Om vi tittar på additionen \(a+1\) har vi per definition att \(a+1=a+S(0)\).
  Från \cref{eq:AdditionRekursion} får vi att \(a+S(0)=S(a+0)\).
  Vi har från \cref{eq:AdditionNoll} att \(a+0=a\).
  Vi får då att
  \begin{equation}\label{eq:NaturligaAdderaEtt}
    a+1 = a+S(0) = S(a+0) = S(a).
  \end{equation}
  Således är \(a+1\) efterföljaren \(S(a)\) till \(a\).
\end{proof}

\begin{example}
  Om vi vidare tittar på additionen \(a+2\) har vi att
  \begin{equation*}
    a+2 = a+S(1) = S(a+1).
  \end{equation*}
  Vi har från \cref{NaturligaAdderaEtt} att \(a+1=S(a)\) och vi får att
  \begin{equation*}
    a+2=S(S(a))=(a+1)+1.
  \end{equation*}
\end{example}

Vi kan nu formulera en vidareutveckling av resultatet 
i \cref{NaturligaAdderaEtt}.
Vi ger vidareutvecklingen genom följande hjälpsats.
\begin{lemma}\label{lem:NaturligaAdderaN}
  Om \(a\) och \(n\) är naturliga tal gäller att
  \begin{equation}
    a+n=S^n(a).
  \end{equation}
\end{lemma}
\begin{proof}
  Vi har enligt \cref{def:NaturligaBeteckningar} att \(n=S^n(0)\).
  Då har vi att \(a+n=a+S^n(0)\).
  Enligt \cref{eq:AdditionRekursion} är \(a+S^n(0) = S(a+S^{n-1}(0))\), där
  \(n-1\) får beteckna det naturliga tal sådant att \(S(n-1)=n\).
  Men enligt \cref{eq:AdditionRekursion} är \(a+S^{n-1}(0)=S(a+S^{n-2}(0))\),
  där \(n-2\) är det naturliga tal sådant att \(S^2(n-2)=n\).
  Vi har nu att \[a+n=S(S(a+S^{n-2}(0)))=S^2(a+S^{n-2}(0)).\]
  Således har vi \(S^k(a+S^{n-k}(0))\) för \(k\leq n\).
  När \(k=n\) får vi \(S^n(a+S^{n-n}(0)) = S^n(a+0) = S^n(a)\).
\end{proof}

\begin{exercise}
  Visa att \(S^a(S^b(0)) = S^{b+a}(0)\).
\end{exercise}

Vi kan nu börja titta på vilka egenskaper som addition har.
En fråga som vi kan ställa oss, spelar det någon roll i vilken ordning vi
adderar?
Spelar det någon roll om vi adderar först \(1\) och \(2\) och sedan adderar
\(3\)?
Följande sats besvarar just den frågan.
\begin{theorem}[Associativitet]\label{thm:NaturligaAssociativitet}\label{thm:NaturligaAdditionAssociativ}\index{associativitet}\index{naturliga 
    tal!associativitet}
  Om \(a,b\) och \(c\) är naturliga tal, då gäller att \(a+(b+c)=(a+b)+c\).
\end{theorem}
\begin{proof}
  Vi börjar med att titta på \(a+(b+c)\).
  Enligt \cref{lem:NaturligaAdderaN} har vi att \(b+c=S^c(b)\).
  Enligt \cref{def:NaturligaBeteckningar} är \(b=S^b(0)\) och vi får
  att \(b+c=S^c(S^b(0))\).
  Vi har då kvar \(a+(b+c)=a+S^c(S^b(0))\).
  Enligt \cref{lem:NaturligaAdderaN} igen har vi
  \begin{equation}\label{eq:NaturligaAssociativ1}
    a+S^c(S^b(0))=S^c(S^b(a))=S^c(S^b(S^a(0))).
  \end{equation}

  Vi fortsätter med att kolla på \((a+b)+c\).
  Då har vi \(a+b=S^b(a)\) och således
  \begin{equation}\label{eq:NaturligaAssociativ2}
    S^b(a)+c=S^c(S^b(a))=S^c(S^b(S^a(0))).
  \end{equation}

  Vi ser att \cref{eq:NaturligaAssociativ1} och
  \cref{eq:NaturligaAssociativ2} är lika och således har vi även att
  \[a+(b+c)=(a+b)+c.\]
\end{proof}

\begin{example}
  Vi vill addera talen \(1,2\) och \(3\) genom \(1+2+3\).
  Enligt \cref{thm:NaturligaAssociativitet} spelar det ingen roll om vi
  först adderar \(1+2=3\) och sedan adderar \(3\), det vill säga \(3+3=6\),
  eller om vi först adderar \(2+3=5\) och sedan adderar \(1+5=6\).
  Som vi ser är \((1+2)+3=3+3=6\) och \(1+(2+3)=1+5=6\) båda lika med \(6\).
\end{example}

Vidare kan vi fråga oss, spelar det någon roll om vi adderar \(1\) med \(2\)
eller om vi adderar \(2\) med \(1\)?
Denna fråga besvaras med följande sats och bevis.
\begin{theorem}[Kommutativitet]\label{thm:NaturligaKommutativitet}\index{kommutativitet}\index{naturliga 
    tal!kommutativitet}
  Om \(a\) och \(b\) är naturliga tal, då gäller att \(a+b=b+a\).
\end{theorem}
\begin{proof}
  Vi har från \cref{lem:NaturligaAdderaN} att \(a+b=S^b(a)=S^b(S^a(0))\).
  Men
  \begin{equation}
    \label{eq:NaturligaKommutativ1}
    S^b(S^a(0)) =
    \underbrace{S(S(S(\cdots(S}_b(\underbrace{S(S(S(\cdots(S}_a(0)))))))))).
  \end{equation}

  På samma sätt har vi att \(b+a=S^a(b)=S^a(S^b(0))\) och
  \begin{equation}
    \label{eq:NaturligaKommutativ2}
    S^a(S^b(0)) =
    \underbrace{S(S(S(\cdots(S}_a(\underbrace{S(S(S(\cdots(S}_b(0)))))))))).
  \end{equation}

  Eftersom att \cref{eq:NaturligaKommutativ1} och
  \cref{eq:NaturligaKommutativ2} är lika måste vi ha \(a+b=b+a\).
\end{proof}

\begin{example}
  Vi vill addera talen \(1\) och \(2\).
  Enligt \cref{thm:NaturligaKommutativitet} spelar det ingen roll om vi 
  gör detta genom \(1+2\) eller \(2+1\).
  I båda fallen kommer vi fram till att \(1+2=2+1=3\).
\end{example}



%%%%%%%%%%%%%%%%%%%%%%%%%%%%%%%%%%%%%
% EGENSKAPER FÖR MULTIPLIKATION
%%%%%%%%%%%%%%%%%%%%%%%%%%%%%%%%%%%%%
\section{Multiplikationens algebraiska egenskaper}
Precis som för addition undrar vi nu hur multiplikation beter sig.
Spelar det någon roll i vilken ordning vi multiplicerar naturliga tal?
En ytterligare fråga som uppstår nu är dock: hur förhåller sig multiplikation
till addition?
Vi ska börja med att besvara denna fråga, sedan fortsätter vi med att undersöka
associativiteten och kommutativiteten som vi gjorde för addition.
\begin{theorem}[Distributivitet]\label{thm:NaturligaDistributivitet}\index{distributivitet}\index{naturliga 
    tal!distributivitet}
  Om \(a,b\) och \(c\) är naturliga tal gäller att
  \(a\cdot(b+c) = a\cdot b + a\cdot c\) och \((a+b)\cdot c = a\cdot c +
  b\cdot c\).
\end{theorem}
\begin{proof}
  Vi har först från \cref{lem:NaturligaAdderaN} att \(b+c=S^c(S^b(0))\).
  Således får vi att \(a\cdot (b+c) = a\cdot (S^c(S^b(0)))\).
  Från \cref{def:NaturligaMultiplikation} får vi då att
  \begin{align}
    \label{eq:NaturligaDistributivitet}
    \begin{split}
      a\cdot (b+c) &= a\cdot (S^c(S^b(0))) = \underbrace{a+a+\cdots+a}_{c} +
        a\cdot S^b(0) \\
        &= \underbrace{a+a+\cdots+a}_{c}+\underbrace{a+a+\cdots+a}_{b}.
    \end{split}
  \end{align}
  Om vi tittar på de enskilda delarna av uttrycket längst till höger.
  Då har vi enligt definitionen för produkten att
  \begin{align*}
    \underbrace{a+a+\cdots+a}_{b}
      &= \underbrace{a+a+\cdots+a}_{b-1}+a\cdot S(0) \\
      &= \underbrace{a+a+\cdots+a}_{b-2}+a\cdot S(1) \\
      &= \underbrace{a+a+\cdots+a}_{b-3}+a\cdot S(2) \\
      & \vdots \\
      &= a\cdot b.
  \end{align*}
  Vi har på samma sätt att
  \begin{equation*}
    \underbrace{a+a+\cdots+a}_{c} = a\cdot c.
  \end{equation*}
  Om vi använder detta i \cref{eq:NaturligaDistributivitet} får vi att
  \begin{equation*}
    a\cdot (b+c) = \underbrace{a+a+\cdots+a}_b+\underbrace{a+a+\cdots+a}_c
      = a\cdot b + a\cdot c,
  \end{equation*}
  vilket visar första delen av satsen.

  Om vi tittar på \((a+b)\cdot c\) får vi att
  \begin{equation*}
    (a+b)\cdot c = \underbrace{(a+b)+(a+b)+\cdots+(a+b)}_c.
  \end{equation*}
  Eftersom att additionen är associativ och kommutativ kan vi byta plats
  på termerna och får då
  \begin{equation*}
    \underbrace{(a+b)+(a+b)+\cdots+(a+b)}_c =
    \underbrace{a+a+\cdots+a}_c+\underbrace{b+b+\cdots+b}_c.
  \end{equation*}
  På samma sätt som ovan får vi att detta är \(a\cdot c + b\cdot c\).
  Då har vi visat att \((a+b)\cdot c = a\cdot c + b\cdot c\) och vi har visat
  satsen.
\end{proof}

\begin{exercise}
  Gäller det också att
  \(a\cdot (b_1+b_2+\cdots+b_n) =
    a\cdot b_1 + a\cdot b_2 + \cdots + a\cdot b_n\)?
\end{exercise}

Vi vet nu hur multiplikationen förhåller sig till additionen och kan då gå
vidare till att undersöka om multiplikationen har de associativa och
kommutativa egenskaperna som additionen har.
Vi börjar med associativiteten.
\begin{theorem}[Associativitet]\label{thm:NaturligaMultiplikationAssociativ}\index{associativitet}\index{naturliga 
    tal!associativitet}
  Om \(a\) och \(b\) är naturliga tal, då gäller att \(a\cdot(b\cdot
  c)=(a\cdot b)\cdot c\).
\end{theorem}
\begin{proof}
  Vi tittar först på \(a\cdot (b\cdot c)\) och får enligt definitionen för
  multiplikation att
  \begin{equation*}
    a\cdot (b\cdot c) = a\cdot (\underbrace{b+b+\cdots+b}_c).
  \end{equation*}
  Eftersom att multiplikationen är distributiv
  (\cref{thm:NaturligaDistributivitet}) får vi att
  \begin{equation}
    \label{eq:NaturligaMultAssoc1}
    a\cdot (\underbrace{b+b+\cdots +b}_c) = \underbrace{a\cdot b+a\cdot
      b+\cdots+a\cdot b}_c.
  \end{equation}

  Vi tittar nu på \((a\cdot b)\cdot c\).
  Enligt definitionen för multiplikation är
  \begin{equation}
    \label{eq:NaturligaMultAssoc2}
    (a\cdot b)\cdot c = \underbrace{a\cdot b+a\cdot b+\cdots+a\cdot b}_c.
  \end{equation}

  Eftersom att \cref{eq:NaturligaMultAssoc1} och
  \cref{eq:NaturligaMultAssoc2} är lika har vi visat satsen.
\end{proof}

\begin{theorem}[Kommutativitet]\index{kommutativitet}\index{naturliga 
    tal!kommutativitet}
  Om \(a\) och \(b\) är naturliga tal, då gäller att \(a\cdot b=b\cdot a\).
\end{theorem}
\begin{proof}
  Beviset använder induktion.
  Låt \(a\) vara ett naturligt tal.
  Vi vill visa att \(a\cdot b = b\cdot a\) för alla naturliga tal \(b\).
  För \(b=0\) är det klart att multiplikationen är kommutativ.
  För \(b=1\) har vi att
  \begin{equation}
    \label{eq:MultKommutativ1}
    a\cdot 1 = a+a\cdot 0 = a.
  \end{equation}
  Vi har också att
  \begin{equation*}
    1\cdot a = \underbrace{1+1+\cdots+1}_a.
  \end{equation*}
  Enligt \cref{NaturligaAdderaEtt} och att additionen är associativ
  (\cref{thm:NaturligaAdditionAssociativ}) får vi att
  \begin{equation}
    \label{eq:MultKommutativ2}
    \underbrace{1+1+\cdots+1}_a = S^a(0) = a.
  \end{equation}
  Eftersom att \cref{eq:MultKommutativ1} och \cref{eq:MultKommutativ2} är
  lika måste också \(a\cdot 1 = 1\cdot a\).

  Antag att multiplikationen är kommutativ för alla \(b\) mindre än \(k\).
  Vi har då att \(a\cdot k = k\cdot a\).
  Vi vill nu visa att då måste multiplikationen vara kommutativ även för
  \(b=k+1\).
  Eftersom att multiplikation är distributiv över addition
  (\cref{thm:NaturligaDistributivitet}) har vi att \(a\cdot (k+1) = a\cdot
  k + a\cdot 1\).
  Vi har redan konstaterat att \(a\cdot k = k\cdot a\) och att \(a\cdot 1 =
  1\cdot a\), och följaktligen är \(a\cdot (k+1) = a\cdot k + a\cdot 1 =
  k\cdot a + 1\cdot a\).
  Vi har från distributiviteten igen att \(k\cdot a + 1\cdot a = (k+1)\cdot
  a\).
  Således är multiplikationen kommutativ även för \(b=k+1\) och den måste
  därför vara kommutativ för alla naturliga tal.
\end{proof}
\begin{exercise}
  Vilka likheter finns mellan beviset ovan och induktionsaxiomet för de
  naturliga talen?
\end{exercise}

\begin{exercise}
  Visa att \((a+b)(c+d) = ac+ad+bc+bd\).
\end{exercise}


%%%%%%%%%%%%%%%%%%%%%%%%%%%%%%%%%%%
% ALGEBRAISKA EGENSKAPER
%%%%%%%%%%%%%%%%%%%%%%%%%%%%%%%%%%%
\section{Algebraiska egenskaper för de naturliga talen}%
\label{sec:HeltalensAlgebraiskaEgenskaper}
Det är nu dags att sammanfatta de algebraiska egenskaperna för de
naturliga talen.
Vi har i tidigare avsnitt visat att de naturliga talen har följande egenskaper.

\theoremstyle{plain}
\newtheorem*{AlgebraicPropertiesNatural}{Algebraiska egenskaper för de
  naturliga talen}
\begin{AlgebraicPropertiesNatural}\label{def:HeltalenEgenskaper}
  På mängden \(\N\) av hela tal definieras två binära operationer,
  addition (\(+\)) och multiplikation (\(\cdot\)).
  För addition gäller följande:
  \begin{description}
    \item[Kommutativitet] \(a+b=b+a\) för alla \(a,b\in\N\).
    \item[Associtivitet] \((a+b)+c=a+(b+c)\) för alla \(a,b,c\in\N\).
    \item[Additivt identitetselement] Det finns ett element \(0\in\N\)
      sådant att för alla \(a\in\N\) gäller att \(0+a = a+0 = a\).
  \end{description}
  För multiplikation gäller följande:
  \begin{description}
    \item[Kommutativitet] \(a \cdot b=b \cdot a\) för alla \(a,b\in\N\).
    \item[Associtivitet] \((a \cdot b) \cdot c=a \cdot (b \cdot c)\) för
      alla \(a,b,c\in\N\).
    \item[Multiplikativt identitetselement] Det finns ett element
      \(1\in\N\) sådant att för alla \(a\in\N\) gäller att
      \(1 \cdot a = a \cdot 1 = a\).
  \end{description}
  Utöver detta gäller även
  \begin{description}
    \item[Multiplikativ distributivitet över addition]
      \(a \cdot (b+c) = (a \cdot b) + (a \cdot c)\) och
      \((b+c) \cdot a = (b \cdot a) + (c \cdot a)\) för alla reella tal
      \(a,b,c\in\N\).
  \end{description}
\end{AlgebraicPropertiesNatural}

För en djupare och mer generell behandling av dessa egenskaper rekommenderas 
läsaren till~\cite{Bartle2000itr,Grillet2007aa}.


%%%%%%%%%%%%%%%%%%%%%%%%%%%%%%%%%%%%%
% POTENSER
%%%%%%%%%%%%%%%%%%%%%%%%%%%%%%%%%%%%%
\section{Potenser}
Det kan bli tröttsamt i längden att skriva ut många faktorer i en
produkt.
För additionen kunde vi istället för att skriva \(a+a+\cdots+a\)
multiplicera \(a\) med antalet \(a\)-termer i summan, på så vis behöver vi inte
skriva ut alla \(a\)-termer utan det räcker med att vi vet vilken term och hur
många vi ska addera.
Vi vill naturligtvis kunna göra liknande för multiplikation.
Istället för att skriva ut alla \(b\)-faktorer i en produkt \(b\cdot
b\cdots b\) räcker det med att vi skriver faktorn och antalet av denna
faktor.
Då skriver vi produkten \(b\cdot b\cdots b\) med \(n\) faktorer som \(a^n\).
Detta åstadkommer vi med potenser som definieras enligt följande.
\begin{definition}
  Låt \(a\) och \(n\) vara naturliga tal.
  Låt \(a^1=a^{S(0)}=a\).
  Om \(a^n\) är definierad låter vi \(a^{S(n)}=a\cdot a^n\).
  Vi kallar \(a^n\) för en \emph{\(a\)-potens} med \emph{exponenten} \(n\).
\end{definition}
\begin{remark}
  Notera att \(a^0\) ej är definierad för något naturligt tal \(a\).
  Vi återkommer till detta i \cref{ch:Rationella} som handlar om rationella
  tal.
\end{remark}

\begin{example}
  Vi har produkten \(2\cdot 2\cdot 2\) som består av tre termer som alla är
  \(2\).
  Vi kan då skriva produkten som en \(2\)-potens, nämligen \(2^3\).
  Enligt definitionen är \(2^3=2\cdot 2^2=2\cdot 2\cdot 2^1 = 2\cdot 2\cdot
  2\).
\end{example}

\begin{example}
  Talet \(72\) kan skrivas som produkten \(8\cdot 9=2\cdot 2\cdot 2\cdot
  3\cdot 3\).
  Om vi använder potensform får vi att \(72=2^3\cdot 3^2=2^3 3^2\).
\end{example}

\begin{example}
  Talet \(4\) kan skrivas som en \(4\)-potens, nämligen \(4^1\).
  Det kan också skrivas som en \(2\)-potens, nämligen \(4=2\cdot 2=2^2\).
\end{example}


%%%%%%%%%%%%%%%%%%%%%%%%%%%%%%%%%%%%%
% RESULTAT OM POTENSER
%%%%%%%%%%%%%%%%%%%%%%%%%%%%%%%%%%%%%
\subsection{Några resultat om potenser}
Vi ska nu titta på några enkla resultat som följer av vår definition av
potenser.
En första fråga kan vara, vad händer om vi adderar två potenser?
\begin{example}\label{ex:AdderaPotenser}
  Vi vill addera potenserna \(a^n\) och \(b^m\) där \(a\) och \(b\) är
  naturliga tal och \(n\) och \(m\) är naturliga tal skilda från noll.
  Om vi adderar dem får vi
  \begin{equation*}
    a^n + b^m = \underbrace{a\cdot a\cdots a}_n + \underbrace{b\cdot
    b\cdots b}_m.
  \end{equation*}
  Vi kan inte komma särskilt mycket längre och vi kan konstatera att detta
  var ett föga intressant resultat.
\end{example}

Om vi istället testar att multiplicera två potenser.
\begin{example}\label{ex:MultipliceraPotenser}
  Vi vill multiplicera potenserna \(a^n\) och \(b^m\) där \(a\) och \(b\) är
  naturliga tal och \(n\) och \(m\) är naturliga tal skilda från noll.
  Vi får då att
  \begin{equation*}
    a^n\cdot b^m = a^n b^m = \underbrace{a\cdot a\cdots a}_n\cdot
    \underbrace{b\cdot b\cdots b}_m.
  \end{equation*}
  Detta verkar mer lovande, vad händer om \(a=b\)?

  Om \(a=b\) får vi att
  \begin{multline}
    \label{eq:MultipliceraPotenser}
    a^n b^m = a^n a^m = \underbrace{a\cdot a\cdots a}_n\cdot\underbrace{a\cdot
    a\cdots a}_m = \\
    \underbrace{a\cdot a\cdots a}_{n}\cdot a^m = a^{S^n(m)} = a^{n+m}.
  \end{multline}
\end{example}

Detta är ett intressant resultat och vi sammanfattar det som följande sats.
\begin{theorem}\label{thm:AdderaExponenter}
  Om \(a\), \(n\neq 0\) och \(m\neq 0\) är naturliga tal, då är \(a^n a^m =
  a^{n+m}\).
\end{theorem}
\begin{proof}
  Satsen följer direkt från \cref{eq:MultipliceraPotenser}.
\end{proof}

\begin{example}
  Vi vill multiplicera potenserna \(2^3\) och \(2^5\).
  Vi får då enligt \cref{thm:AdderaExponenter} att
  \begin{equation*}
    2^3 2^5 = 2^{3+5} = 2^8.
  \end{equation*}
  Om vi kontrollerar ser vi att \(2^3=8\) multiplicerat med \(2^5=32\)
  faktiskt är \(2^8=256\).
\end{example}

Eftersom att det fungerar att addera exponenterna ter det sig naturligt att
fråga om vi även kan multiplicera exponenterna och vad det skulle betyda.
\begin{example}\label{ex:MultipliceraExponenter}
  Om \(a\), \(n\neq 0\) och \(m\neq 0\) är naturliga tal kan vi skapa
  potensen \(a^{nm}\).
  Vi vet att \[nm = \underbrace{n+n+\cdots+n}_m\] och således att
  \begin{equation}
    \label{eq:MultipliceraExponenter1}
    a^{nm}=a^{n+n+\cdots+n}=\underbrace{a^n a^n\cdots a^n}_m.
  \end{equation}
  Men vi har nyss definierat potenser för att förenkla skrivandet av sådana
  produkter.
  Följaktligen får vi att
  \begin{equation}
    \label{eq:MultipliceraExponenter2}
    a^n a^n\cdots a^n = (a^n)^m.
  \end{equation}
\end{example}
Vi sammanfattar resultatet i följande sats.
\begin{theorem}
  Låt \(a\), \(n\neq 0\) och \(m\neq 0\) vara naturliga tal.
  Då gäller att \((a^n)^m = a^{nm}\).
\end{theorem}
\begin{proof}
  Satsen följer från \cref{ex:MultipliceraExponenter}.
\end{proof}

Vi har nu visat att vi har addition och multiplikation av exponenter.
Följande sats visar också att multiplikation av exponenter är distributiv över
addition av exponenter.
\begin{theorem}\label{thm:MultipliceraExponenter}
  Om \(a\), \(b\), \(n\neq 0\), \(m\neq 0\) och \(p\neq 0\) är naturliga tal,
  då är \((a^n b^m)^p = a^{np} b^{mp}\).
\end{theorem}
\begin{proof}
  Om vi tittar på vad \((a^n b^m)^p\) faktiskt betyder, så finner vi att
  \begin{equation*}
    (a^n b^m)^p = \underbrace{(a^n b^m)(a^n b^m)\cdots (a^n b^m)}_p.
  \end{equation*}
  Eftersom att multiplikation är associativ och kommutativ får vi att
  \begin{equation*}
    \underbrace{(a^n b^m)(a^n b^m)\cdots (a^n b^m)}_p =
    \underbrace{a^n a^n\cdots a^n}_p \underbrace{b^m b^m\cdots b^m}_p =
    (a^n)^p (b^m)^p.
  \end{equation*}
  Vi har från \cref{ex:MultipliceraExponenter} att \((a^n)^p(b^m)^p =
  a^{np}b^{mp}\).
  Och därför är \((a^n b^m)^p = a^{np} b^{mp}\).
\end{proof}

Vi avslutar avsnittet med ett exempel som nyttjar de båda satserna.
\begin{example}
  Vi vill multiplicera \(4^3\) och \(12^2\).
  Vi vet att \(4=2\cdot 2\), det vill säga \(4=2^2\), samt att \(12=3\cdot
  4=3\cdot 2^2\).
  Om vi använder detta och tittar på vad vi hade från början, \(4^3\) är
  således \((2^2)^3=2^{2\cdot 3}=2^6\).
  \(12^2\) blir då \((3\cdot 2^2)^2\) och således \(12^2=3^{1\cdot
  2} 2^{2\cdot 2} = 3^2 2^4\).
  Om vi multiplicerar dem får vi
  \begin{equation*}
    (\underbrace{2^6}_{4^3}\cdot \underbrace{3^2 2^4}_{12^2}) =
    2^6 2^4 3^2 = 2^{6+4} 3^2 = 2^{10} 3^2.
  \end{equation*}
\end{example}

\begin{exercise}
  Visa att addition av exponenter och multiplikation av exponenter båda är
  associativa och kommutativa operationer.
\end{exercise}


%%%%%%%%%%%%%%%%%%%%%%%%%%%%%%%%%%%%%
% VÄLORDNINGSPRINCIPEN
%%%%%%%%%%%%%%%%%%%%%%%%%%%%%%%%%%%%%
\subsection{Välordningsprincipen}
\dots

\begin{theorem}[Välordningsegenskapen hos de naturliga 
  talen]\label{def:Valordningsprincipen}\index{välordningsegenskapen}\index{naturliga 
    tal!välordningsegenskapen}
  Om en mängd \(M\subseteq\N\) är en delmängd till de naturliga talen och om
  \(M\neq\emptyset\) ej är den tomma mängden, då existerar ett element
  \(m\in M\) i \(M\) sådant att \(m\leq k\) är mindre än \(k\) för alla
  \(k\in M\) i \(M\).
\end{theorem}
\begin{proof}
  \dots
\end{proof}


%%%%%%%%%%%%%%%%%%%%%%%%%%%%%%%%%%%%%
% AVSLUTANDE REFLEKTION
%%%%%%%%%%%%%%%%%%%%%%%%%%%%%%%%%%%%%
\section{Avslutande reflektion}
Som en avslutande reflektion till detta kapitel ges följande
övningsuppgifter.
Tanken med dessa övningar är att reflektera över matematikens existens, hur den
förhåller sig till verkligheten etc.

\begin{exercise}
  Diskutera förhållandet mellan matematiken och verkligheten.
  En inledande fråga i diskussionen kan vara:
  Grundar sig matematiken i verkligheten eller är den oberoende av
  verkligheten?
  Diskutera.
\end{exercise}

\begin{exercise}
  a: Om matematiken är oberoende av verkligheten, hur kan den användas
  för att utforska och förutsäga verkligheten? Och om verkligheten såg
  annorlunda ut, skulle matematiken se likadan ut?

  b: Om matematiken grundar sig i verkligheten, hur kan den användas för att
  förutsäga verkligheten när den behöver verkligheten för att visas vara sann?
\end{exercise}

