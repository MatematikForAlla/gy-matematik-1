% Copyright 2016; Daniel Bosk <daniel@bosk.se>
%
% This work is licensed under the Creative Commons Attribution-ShareAlike 4.0 
% Unported license.  To view a copy of this license, visit URL
%
%   http://creativecommons.org/licenses/by-sa/4.0/.
%

\chapter{Sannolikhetsteori}
\label{Sannolikhet}
% XXX Write introduction to probability theory chapter
% This should provide a historical overview of area.
\dots


\section{Utfall, händelser och sannolikhetsmått}

\begin{definition}[Utfall]\label{Utfall}\index{utfall}
  Ett \emph{utfallsrum} är en uppräknelig icke-tom mängd.
  Elementen i utfallsrummet kallas \emph{utfall}.
  En \emph{händelse} är en delmängd till utfallsrummet.
\end{definition}

Exempelvis, utfallsrummet för ett tärningskast skulle vara
\begin{equation}\label{UtfallsrumTärningskast}
  \Omega_T = \{\fcdice{1}, \fcdice{2}, \fcdice{3}, \fcdice{4}, \fcdice{5}, 
  \fcdice{6}\},
\end{equation}
där varje tärningssymbol representerar att tärningen landar med den sidan upp.
En händelse är enligt \cref{Utfall} en delmängd till \(\Omega_T\).
Det vill säga mängden av händelser för utfallsrummet \(\Omega_T\) är 
potensmängden \(\powerset(\Omega_T)\).
Detta betyder som kanske väntat att \(\{\fcdice{1}\},\ldots,\{\fcdice{6}\}\) är 
händelser.
Det är kanske inte lika väntat att \(\emptyset\) och \(\{\fcdice{1}, \ldots, 
  \fcdice{6}\}\) också är händelser.
En händelse betyder alltså att något av utfallen sker, exempelvis för att vi 
ska kunna uttrycka att \enquote{tärningen faller med en jämn sida upp} --- 
vilket motsvaras av händelsen \(\{\fcdice{2}, \fcdice{4}, \fcdice{6}\}\), det 
vill säga att något av utfallen sker.

\begin{exercise}\label{NegationAvHändelse}
  Utforska huruvida det går att uttrycka negationen av en händelse.
  Exempelvis händelsen \enquote{slå en sexa} och dess negation \enquote{slå 
    inte en sexa}, eller \enquote{slå ett jämnt tal} och dess negation 
  \enquote{slå inte en jämnt tal}.
  Skulle \enquote{slå ett udda tal} motsvara negationen av \enquote{slå ett 
    jämnt tal}?
\end{exercise}

Vi vill nu fortsätta med att formalisera sannolikheter för att händelser 
inträffar.

\begin{definition}[Sannolikhetsmått]\label{Sannolikhetsmått}\index{sannolikhetsmått}
  Låt \(\Omega\) vara ett utfallsrum.
  Ett \emph{sannolikhetsmått över utfallsrummet \(\Omega\)} är en funktion 
  \(\Pr\colon \Omega\to [0,1]\) som uppfyller följande:
  \begin{enumerate}
    \item \(\Pr[ \emptyset ] = 0\) och \(\Pr[ \Omega ] = 1\),
    \item\label{SummeraSannolikhetsmatt} Om \(A\subseteq \Omega\) och 
      \(B\subseteq \Omega\) båda är utfall och \(A\cap B = \emptyset\), då är 
      \(\Pr[A\cup B] = \Pr[A] + \Pr[B]\).
  \end{enumerate}
\end{definition}
Denna definition är, likt alla definitioner, av stor vikt.
Vi kommer att bygga alla resultat i kapitlet på denna definition.
Vi börjar med att se ett exempel på ett sannolikhetsmått.

\begin{example}\label{ExempelTärningskast}
  Låt \(\Omega_T\) vara utfallsrummet för ett tärningskast,  
  \cref{UtfallsrumTärningskast}.
  Hur bör vi definiera \(\Pr\colon \powerset(\Omega_T)\to \interval{0}{1}\)?
  För att det ska avspegla vad vi avser med en riktig tärning bör 
  \(\Pr[\{\fcdice{1}\}] = \cdots = \Pr[\{\fcdice{6}\}] = \frac{1}{6}\).
\end{example}

\begin{exercise}
  Visa att \cref{ExempelTärningskast} uppfyller \cref{Sannolikhetsmått}.
\end{exercise}

\begin{exercise}
  Låt \(\Omega_T\) och \(\Pr\) vara definierade som 
  i \cref{ExempelTärningskast}.
  Låt \(J\) vara händelsen att en tärning visar ett jämnt tal efter ett kast.
  Sannolikheten att detta händer är \(\Pr[J] = \frac{3}{6}\), varför?
\end{exercise}

En kombination av utfallsrum och sannolikhetsmått, som vi har ovan, kallar vi 
för ett sannolikhetsrum.

\begin{definition}[Sannolikhetsrum]\label{Sannolikhetsrum}\index{sannolikhetsrum}
  Om \(\Omega\) är ett utfallsrum och \(\Pr\colon \Omega\to \interval{0}{1}\) 
  är ett sannolikhetsmått över
  \(\Omega\), då kallar vi \((\Omega, \Pr)\) för ett \emph{sannolikhetsrum}.
\end{definition}


\section{Att kombinera händelser}

Vi ska nu utforska några intressanta resultat om sannolikhetsrum generellt, det 
vill säga vad våra definitioner ovan ger upphov till.
Låt oss börja med ett lemma som följer direkt av \cref{Sannolikhetsmått}.

Vi har ett sannolikhetsrum \((\Omega, \Pr)\) och är intresserade av \(\Pr[A]\), 
där \(A\subset \Omega\).
Ibland känner vi inte till exakt vad \(\Pr[A]\) är, men vi känner till 
\(\Pr[B_1] = p_1, \dotsc, \Pr[B_n] = p_n\) för \(B_i\subset \Omega\), \(i = 1, 
  \dotsc, n\).
Om \(A = B_1\cup \dotsb\cup B_n\) och \(B_1\cap \dotsb\cap B_n = \emptyset\), 
då kan vi enligt \cref{Sannolikhetsmått} beräkna \(\Pr[A] = p_1 + \dotsb 
  + p_n\).
Vi ska nu sammanfatta några sätt att beräkna sannolikheten för en händelse.

\begin{lemma}[Sannolikheten för en händelse]\label{SannolikhetHändelse}
  Låt \((\Omega, \Pr)\) vara ett sannolikhetsrum och \(A\subseteq \Omega\) en 
  händelse.
  Då kan vi beräkna \(\Pr[A]\) på följande sätt:
  \begin{enumerate}
    \item \(\Pr[A] = \sum_{a\in A} \Pr[\{a\}]\), det vill säga summan av alla 
      utfall i händelsen;
    \item \(\Pr[A] = \sum_{B\in C} \Pr[B]\), där \(\bigcup C = A\) och 
      \(\bigcap C = \emptyset\).
  \end{enumerate}
\end{lemma}

\begin{exercise}
  Bevisa \cref{SannolikhetHändelse}.
  (Det kan göras genom att använda \cref{Sannolikhetsmått} och induktion.)
\end{exercise}

\begin{exercise}
  Utforska vilken typ av händelser vi kan uttrycka och beräkna sannolikheten 
  för, och vad vi behöver för att kunna göra beräkningen.
\end{exercise}

Vi ska nu återkomma till en händelse och dess negation 
(\cref{NegationAvHändelse}).
Ibland händer det att vi är intresserade av negationen av en händelse, ofta 
händer det också att det är enklare att uttrycka negationen av negationen av en
händelse.
Vi ska nu titta på hur vi kan använda detta för att beräkna sannolikheter.

\begin{theorem}[Sannolikhet för händelse och 
  komplement]\label{SannolikhetKomplement}
  Låt \((\Omega, \Pr)\) vara ett sannolikhetsrum.
  Låt \(A\subseteq \Omega\) vara en händelse och \(B = \Omega\setminus A\) dess 
  komplement.
  Då är \(\Pr[B] = 1 - \Pr[A]\).
\end{theorem}

\begin{proof}
  \Cref{Sannolikhetsmått} sa att \(\Pr[\Omega] = 1\).
  Men \(A\cup B = \Omega\), således måste \(\Pr[A\cup B] = 1\).
  \Cref{Sannolikhetsmått} sa även att för \(A\subseteq \Omega\) och 
  \(B\subseteq \Omega\) där \(A\cap B = \emptyset\), vilket är fallet för 
  komplement, har vi att \(\Pr[A\cup B] = \Pr[A] + \Pr[B]\).
  Alltså är även \(\Pr[A\cup B] = \Pr[A] + \Pr[B]\).
  Följaktligen är \(\Pr[A] + \Pr[B] = 1\) vilket ger oss \(\Pr[B] 
    = 1 - \Pr[A]\).
\end{proof}

\Cref{SannolikhetKomplement} är en väldigt användbar sats då den enkelt kan 
användas för att negera händelser.
Om vi har utfallsrummet \(\Omega\) och händelsen \(A\), då är komplementet till 
\(A\), det vill säga \(B = \Omega\setminus A\), händelsen att \(A\) inte 
inträffar.

\begin{example}[Att inte slå en sexa]
  Vi har utfallsrummet \(\Omega_T\) från \cref{UtfallsrumTärningskast}.
  Låt \(A\) vara händelsen att vi slår en sexa, det vill säga \(A 
    = \{\fcdice{6}\}\).
  Om vi inte slår en sexa måste det betyda att vi slår någon av de andra 
  sidorna, det vill säga \(B = \{\fcdice{1}, \fcdice{2}, \fcdice{3}, 
    \fcdice{4}, \fcdice{5}\}\).
  Vi kan se att \(B = \Omega\setminus A\).
  Vi kan också se att \(\Pr[A] = \frac{1}{6}\).
  Då återstår att se vad \(\Pr[B]\) är.
  Det framgår från \cref{Sannolikhetsmått} att \(\Pr[B] = \Pr[\{\fcdice{1}\}] 
    + \Pr[B\setminus \{\fcdice{1}\}]\).
  \(\Pr[B\setminus \{\fcdice{1}\}]\) kan vi i sin tur dela upp som 
  \(\Pr[B\setminus \{\fcdice{1}\}] = \Pr[\{\fcdice{2}\}] + \Pr[B\setminus 
    \{\fcdice{1}, \fcdice{2}\}]\).
  Fortsätter vi så får vi att \(\Pr[B] = \sum_{t\in B} \Pr[\{t\}] 
    = \frac{5}{6}\).
  Använder vi \cref{SannolikhetKomplement} får vi \(\Pr[B] = 1 - \Pr[A] 
    = \frac{6}{6} - \frac{1}{6} = \frac{5}{6}\).
\end{example}

\begin{exercise}
  I logiken har vi operationerna \(\lnot, \lor, \land\) (negation, disjunktion 
  respektive konjunktion).
  \Cref{SannolikhetKomplement} beskriver hur vi kan beräkna sannolikheten för 
  negationen av en händelse.
  Kan du finna motsvarigheterna för de andra operationerna (disjunktion och 
  konjunktion)?
\end{exercise}

Vi har hittills sett att \(\Pr[A\cup B] = \Pr[A] + \Pr[B]\) förutsatt att 
\(A\cap B = \emptyset\), vi ska försöka att eliminera kravet att \(A\cap
  B =\emptyset\) i nästa avsnitt.
Innan vi fortsätter dit ska vi dock titta på vad \(\Pr[A\cap B]\) skulle kunna 
betyda.

\begin{theorem}[Sannolikhet för två händelser]
  \(\Pr[A, B] = \Pr[A]\cdot \Pr[B|A]\)
\end{theorem}

\begin{proof}
  \dots
\end{proof}

% XXX Maybe move this to the chapter on set theory?
%\begin{definition}[Parvis 
%  disjunkta]\label{ParvisDisjunkta}\index{disjunkt!parvis}
%  Låt \(M_1, \ldots, M_n\) vara \(n\in \N\) mängder.
%  Vi säger att mängderna är \emph{parvis disjunkta} om \(m_i\cap m_j 
%    = \emptyset\) för varje \(1\leq i, j\leq n\) med \(i\neq j\).
%\end{definition}


\section{Betingad sannolikhet}

Oberoende och beroende händelser \dots

\begin{definition}[Betingad sannolikhet]
  \(\Pr[B|A] = \frac{\Pr[B\cap A]}{\Pr[A]}\)
\end{definition}

\begin{definition}[Oberoende händelser]
  Låt \((\Omega, \Pr)\) vara ett sannolikhetsrum.
  Två händelser \(A\subseteq \Omega\) och \(B\subseteq \Omega\) sägs vara 
  \emph{oberoende} om \(\Pr[A\cap B] = \Pr[A]\cdot \Pr[B]\).
\end{definition}

% TRANSLATE and REWORK %%%%%%%%%%%%%%%%%%%%%%%%%%%

\begin{example}
  \begin{itemize}
    \item We roll one die, \(\Omega_D = \{\fcdice{1}, \ldots, \fcdice{6}\}\).
    \item Events: roll a one, \(A = \{\fcdice{1}\}\), and roll a six, \(B 
        = \{\fcdice{6}\}\).

    \item Clearly we cannot get \fcdice{1} and \fcdice{6} at the same time:
      \[\Pr[A\cap B] = \Pr[\emptyset] = 0 \neq \Pr[A]\cdot \Pr[B].\]

    \item But we can get either: \(\Pr[A\cup B] = \Pr[A] + \Pr[B] 
        = \frac{2}{6}\).
  \end{itemize}
\end{example}

\begin{example}
  \begin{itemize}
    \item We roll two dice: \(\Omega_{D2} = \{(\fcdice{1},\fcdice{1}), 
        \ldots, (\fcdice{1}, \fcdice{6}), \ldots, (\fcdice{6}, \fcdice{1}), 
        \ldots, (\fcdice{6}, \fcdice{6})\).

    \item Roll a one: \(A = \{(\fcdice{1}, \fcdice{1}), \ldots, (\fcdice{1}, 
        \fcdice{6}), (\fcdice{2}, \fcdice{1}), \ldots, (\fcdice{6}, 
        \fcdice{1})\}\).

    \item Roll a six: \(B = \{(\fcdice{6}, \fcdice{1}), \ldots, (\fcdice{6}, 
        \fcdice{6}), (\fcdice{1}, \fcdice{6}), \ldots, (\fcdice{5}, 
        \fcdice{6})\}\).

    \item Roll a one and a six: \(\Pr[A\cap B] = \Pr[\{(\fcdice{1}, 
        \fcdice{6}), (\fcdice{6}, \fcdice{1})\}] = \frac{2}{36} \neq 
        \frac{1}{6}\cdot \frac{1}{6}\).

    \item Roll a one or a six with two chances: \(\Pr[A\cup B]\).

  \end{itemize}
\end{example}

\begin{example}
  \begin{itemize}
    \item We roll two dice: \(\Omega_{D2} = \{(\fcdice{1},\fcdice{1}), 
        \ldots, (\fcdice{1}, \fcdice{6}), \ldots, (\fcdice{6}, \fcdice{1}), 
        \ldots, (\fcdice{6}, \fcdice{6})\).

    \item \emph{First} roll a one: \(A = \{(\fcdice{1}, \fcdice{1}), \ldots, 
        (\fcdice{1}, \fcdice{6})\}\).

    \item \emph{Second} roll a six: \(B = \{(\fcdice{1}, \fcdice{6}), \ldots, 
        (\fcdice{6}, \fcdice{6})\}\).

    \item First roll a one, then a six: \(\Pr[A\cap B] = \Pr[\{(\fcdice{1}, 
        \fcdice{6})\}] = \frac{1}{36} = \frac{1}{6}\cdot \frac{1}{6}\).

    \item Either roll a one first \emph{or} a six on the second: \(\Pr[A\cup 
        B]\).

  \end{itemize}
\end{example}

\begin{exercise}
  Even if \(\Pr[A]\cdot \Pr[B] = \Pr[A\cap B]\) they are not necessarily 
  independent, why?
\end{exercise}

\subsection{Conditional probability}

\begin{definition}[Conditioned probability]
  \begin{itemize}
    \item Let \(A, B\) be events in probability space \((\Omega, \Pr)\).
    \item The probability of \(B\) given \(A\) is defined as \(\Pr[B\mid A] 
        = \frac{\Pr[B\cap A]}{\Pr[A]}\)
  \end{itemize}
\end{definition}

\begin{example}
  \begin{itemize}
    \item Use probability space for rolling a die, \((\Omega_D, \Pr)\).
    \item Event \(E = \{\fcdice{2}, \fcdice{4}, \fcdice{6}\}\) for even side 
      up.

    \item Let \(A = \{\fcdice{2}\}\) and \(B = \{\fcdice{3}\}\), then
      \[\Pr[A\mid E] = \frac{\Pr[\{\fcdice{2}\}]}{\Pr[E]} = \frac{1}{3}
        \text{ and }
        \Pr[B\mid E] = \frac{\Pr[\emptyset]}{\Pr[E]} = 0.\]

  \end{itemize}
\end{example}

\begin{theorem}[Joint probability]
  \begin{itemize}
    \item Let \(A, B\) be events in probability space \((\Omega, \Pr)\).
    \item Then \(\Pr[A\cap B] = \Pr[A]\cdot \Pr[B\mid A]\)
  \end{itemize}
\end{theorem}

\begin{exercise}
  Using conditional probability we can show this new result, how?
\end{exercise}
%  \begin{proof}
%    \begin{itemize}
%      \item We know from definition: \(\Pr[B\mid A] = \frac{\Pr[A\cap 
%            B]}{\Pr[A]}\).
%
%      \item Thus \(\Pr[B\mid A]\cdot \Pr[A] = \Pr[A\cap B]\).
%    \end{itemize}
%  \end{proof}

\begin{theorem}[Bayes sats]
  \begin{itemize}
    \item Let \(A, B\) be events in probability space \((\Omega, \Pr)\).
    \item Then \(
        \Pr[A\mid B] = \frac{\Pr[B\mid A]\Pr[A]}{\Pr[B]}.
      \)
  \end{itemize}
\end{theorem}

\begin{proof}
  \begin{itemize}
    \item \(\Pr[A\mid B] = \frac{\Pr[A\cap B]}{\Pr[B]}\) and \(\Pr[B\mid A] 
        = \frac{\Pr[A\cap B]}{\Pr[A]}\).

    \item Thus \(\Pr[A\cap B] = \Pr[A\mid B] \Pr[B] = \Pr[B\mid A] \Pr[A]\).

    \item Which means \(\Pr[A\mid B] = \frac{\Pr[B\mid A] \Pr[A]}{\Pr[B]}\).
  \end{itemize}
\end{proof}

\begin{example}[An intrusion detection system]
  \begin{itemize}
    \item Our IDS is this good:
      \begin{itemize}
        \item It detects malicious traffic with \SI{99}{\percent} certainty 
          (true positives).
        \item It discards benign traffic with \SI{99}{\percent} certainty (true 
          negatives).
      \end{itemize}

    \item \emph{Assume} \SI{0.5}{\percent} of \enquote{the traffic} consist of 
      attacks.

    \item Event \(M\) is malicious traffic, \(B\) is benign traffic.
    \item Event \(+\) means IDS declared malicious, \(-\) declared benign.

    \item Then we get \dots
  \end{itemize}
\end{example}

\begin{example}[IDS continued]
  \begin{align*}
    \Pr[M\mid +] &= \frac{\Pr[+\mid M] \Pr[M]}{\Pr[+]}
    = \frac{0.99\cdot 0.005}{\Pr[+]} \\
    &= \frac{0.99\cdot 0.005}{\Pr[+\mid M] \Pr[M] + \Pr[+\mid B] \Pr[B]} 
    \\
    &= \frac{0.99\cdot 0.005}{0.99\cdot 0.005 + 0.01\cdot (1 - \Pr[M])} 
    \\
    &= \frac{0.99\cdot 0.005}{0.99\cdot 0.005 + 0.01\cdot 0.995} \\
    &= 0.33
  \end{align*}
\end{example}

\begin{exercise}
  \begin{itemize}
    \item What's a good-enough probability for \(\Pr[M\mid +]\)?
    \item How do we achieve it?
  \end{itemize}
\end{exercise}


