\chapter{Studiehandledning för Matematik 1c}\label{Studiehandledning}
%\nocite{Tobin2005msl}

\lettrine{K}{ursen Matematik 1c}
är enligt ämnesplanen 100 poäng, det vill säga att den ska ges 100 
lektionstimmar.
Dessa lektionstimmar ska fördelas på de fem huvudområden som ges under rubriken
\emph{Centralt innehåll} i ämnesplanen: 
\begin{itemize}
  \item taluppfattning, aritmetik och algebra;
  \item geometri;
  \item samband och förändring;
  \item sannolikhet och statistik; samt
  \item problemlösning.
\end{itemize}

I planeringen given nedan avser tidsåtgången lärargenomgångar, till exempel
tid när läraren går igenom innehåll vid tavlan.
Denna tid innefattar inte elevers egna arbete, redovisningar av uppgifter,
klassrumsdiskussioner etcetera.

Avsnittet \emph{Problemlösning}, som innehåller punkterna
\begin{itemize}
  \item Strategier för matematisk problemlösning inklusive användning av
    digitala medier och verktyg,
  \item Matematiska problem av betydelse för privatekonomi, samhällsliv och
    tillämpningar i andra ämnen, och
  \item Matematiska problem med anknytning till matematikens kulturhistoria,
\end{itemize}
kan med fördel inkluderas i undervisningen som tar upp övriga områden och
behöver således inte behandlas specifikt.

Inledande i kursen bör vara grundläggande logik\footnote{%
  Från området \emph{Geometri} har vi punkten \enquote{Matematisk argumentation 
    ned hjälp av grundläggande logik inklusive implikation och ekvivalens samt 
    jämförelser med hur man argumenterar i vardagliga sammanhang och inom 
    naturvetenskapliga ämnen}.
}, begreppen definition och axiom samt sats och bevis\footnote{%
  Från området \emph{Geometri} har vi punkten \enquote{Illustration av 
    begreppen definition, sats och bevis, till exempel med Pythagoras sats och 
    triangelns vinkelsumma}.
}.
Detta eftersom att dessa fundamentala begrepp är det som bygger upp
matematiken.
Därefter kan grundläggande mängdlära gås igenom, med axiom och grundläggande
satser, och konstruktionen av de naturliga talen kan visas för att visa hur
matematiken fundamentalt är uppbyggd.
För detaljer se \cref{tbl:inledning}.

\begin{table}
  \caption{%
    Planering för inledningen.
  }\label{tbl:inledning}
  \begin{tabular}{ll}
    Innehåll & Tidsåtgång (timmar) \\
    \toprule
    Logik, definition, sats och bevis. & 2 \\
    \midrule
    Mängder. & 1 \\
    \midrule
    Mängden av naturliga tal. & 1 \\
    \bottomrule
    & 4 \\
  \end{tabular}
\end{table}

Därefter följer lämpligen avsnittet \emph{Taluppfattning, aritmetik
och algebra} eftersom att taluppfattning, aritmetik och algebra är nödvändiga
för vidare studier av matematiken, och alltså nödvändiga för resten av kursen,
samt att det är naturligt att studera heltalen efter de naturliga talen.
För detaljer se \cref{tbl:talteori}.
%Eftersom att avsnittet har fem punkter bör \emph{25 timmar} avsättas för hela
%detta område.

\begin{table}
  \caption{%
    Planering för avsnittet \emph{Taluppfattning, aritmetik och
    algebra}.
  }\label{tbl:talteori}
  \begin{tabular}{ll}
    %\textbf{Innehåll} & \textbf{Tidsåtgång (timmar)} \\
    Innehåll & Tidsåtgång (timmar) \\
    \toprule
    Generalisering av aritmetikens räkneregler till att\\
      hantera algebraiska uttryck. & 2 \\
    \midrule
    Egenskaper hos mängden heltal, olika talbaser samt\\
      begreppen primtal och delbarhet. & 5 \\
    \midrule
    Egenskaper hos mängderna rationella tal, irrationella\\
      tal och reella tal. & 5 \\
    \midrule
    Begreppet linjär olikhet. & 1 \\
    \midrule
    Algebraiska och grafiska metoder för att lösa linjära\\
      ekvationer, olikheter och potensekvationer. & 5 \\
    \midrule
    Metoder för beräkningar inom vardagslivet och\\
      karaktärsämnena med reella tal skrivna på olika\\
      former, inklusive potenser med reella exponenter\\
      samt strategier för användning av digitala verktyg.
      & * \\
    \bottomrule
    & 21 \\
  \end{tabular}
\end{table}

Efter egenskaperna hos heltal, rationella och irrationella tal och algebra kan
funktionsbegreppet introduceras.
Funktionsbegreppet är ett centralt begrepp inom all matematik och bör gås
igenom tidigt i kursen för att kunna användas senare, exempelvis när
de trigonometriska funktionerna och logaritmfunktionen introduceras.
För detaljer se \cref{tbl:funktioner}.

\begin{table}
  \caption{%
    Planering för avsnittet \emph{Samband och förändring}.
  }\label{tbl:funktioner}
  \begin{tabular}{ll}
    Innehåll & Tidsåtgång (timmar) \\
    \toprule
    Begreppen funktion, definitions- och värdemängd samt\\
      egenskaper hos linjära funktioner samt potens-\\
      och exponentialfunktioner. Representationer av\\
      funktioner i form av ord, funktionsuttryck, tabeller\\
      och grafer.  & 6 \\
    \midrule
    Skillnader mellan begreppen ekvation, olikhet, algebraiskt\\
      uttryck och funktion. & 2 \\
    \midrule
    Fördjupning av procentbegreppet: promille, ppm och\\
      procentenheter. & 2 \\
    \midrule
    Begreppen förändringsfaktor och index samt metoder för\\
      beräkning av räntor och amorteringar för olika typer\\
      av lån. & 4 \\
    \bottomrule
    & 14 \\
  \end{tabular}
\end{table}

Då föregående avsnitt \emph{Samband och förändring} avslutas med
procentbegreppet är det lämpligt att fortsätta med detta i avsnittet
\emph{Sannolikhet och statistik}.
Procentbegreppet används då i en annan form än som ett mått av förändring och
eleverna får bygga vidare sin förståelse för hur procentbegreppet kan användas.
Sannolikhetsavsnittet är ett relativt till kursen litet avsnitt, för detaljer
se \cref{tbl:sannolikhet}.

\begin{table}
  \caption{%
    Planering för avsnittet \emph{Sannolikhet och statistik}.
  }\label{tbl:sannolikhet}
  \begin{tabular}{ll}
    Innehåll & Tidsåtgång (timmar) \\
    \toprule
    Begreppen beroende och oberoende händelser samt\\
      metoder för beräkning av sannolikheter vid slumpförsök\\
      i flera steg med exempel från spel och risk- och\\
      säkerhetsbedömningar. & 3 \\
    \midrule
    Granskning av hur statistiska metoder och resultat används\\
      i samhället och i vetenskap. & 3 \\
    \bottomrule
    & 6 \\
  \end{tabular}
\end{table}

Det sista avsnittet i kursen är avsnittet \emph{Geometri}.
Detta avsnitt är ett avskiljt område som har stor historisk betydelse för
matematiken.
I denna presentation kommer funktionsbegreppet att användas vid introduktionen
av de trigonometriska funktionerna, och därför bör geometriavsnittet följa
efter introduktionen av funktionsbegreppet.
För detaljer se \cref{tbl:geometri}.

\begin{table}
  \caption{%
    Planering för avsnittet \emph{Geometri}.
  }\label{tbl:geometri}
  \begin{tabular}{ll}
    Innehåll & Tidsåtgång (timmar) \\
    \toprule
    Pythagoras sats & 1 \\
    \midrule
    Begreppen sinus, cosinus och tangens samt metoder för\\
      beräkning av vinklar och längder i rätvinkliga\\
      trianglar. & 4 \\
    \midrule
    Begreppet vektor och dess representationer såsom riktad\\
      sträcka och punkt i ett koordinatsystem. & 1 \\
    \midrule
    Addition och subtraktion med vektorer, produkten av en\\
      skalär och en vektor samt skalärprodukt av vektorer\\
      i ett koordinatsystem. & 3 \\
    \midrule
    Pythagoras sats i \(N\) dimensioner. & 1 \\
    \bottomrule
    & 10 \\
  \end{tabular}
\end{table}

Avslutningsvis sammanfattas kursplaneringen översiktligt 
i \cref{tbl:sammanfattning}.
Totalt är 55 timmar avsatt för genomgångar, de återstående 45 timmarna är
oplanerade och bör användas till exempelvis elevers egna arbete, redovisning
av uppgifter, klassrumsdiskussioner etcetera.
Dessa timmar skulle kunna spridas jämnt över alla områden, men de bör hellre
nyttjas där eleverna har intresse eller behov av mer tid.

\begin{table}
  \caption{%
    Kursplaneringsöversikt.
  }\label{tbl:sammanfattning}
  \begin{tabular}{ll}
    Innehåll & Tidsåtgång (timmar) \\
    \toprule
    Inledning & 4 \\
    Taluppfattning, aritmetik och algebra & 21 \\
    Samband och förändring & 14 \\
    Sannolikhet och statistik & 6 \\
    Geometri & 10 \\
    \midrule
    & 55 \\
    \bottomrule
    Oplanerad tid & 45 \\
  \end{tabular}
\end{table}

