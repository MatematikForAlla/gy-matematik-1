% $Id$
%
% Copyright 2011, 2012, 2013, 2014, 2015; Daniel Bosk <daniel@bosk.se>
%
% This work is licensed under the Creative Commons Attribution-ShareAlike 4.0 
% Unported license.  To view a copy of this license, visit URL
%
%   http://creativecommons.org/licenses/by-sa/4.0/.
%
\chapter{Introduktion}\label{Introduktion}
% XXX fysik, kemi använder matematiken
\lettrine{M}{atematiken har funnits} i mer än 5000 år, men började utvecklas i 
riktning mot dagens matematik först omkring 300 f.v.t.\ i antikens 
Grekland~\cite{Kline1990mtf1}.
Innan dess var matematiken endast räkning, ett verktyg för att beräkna skatter 
och konstruera byggnadsverk.

Ordet matematik har enligt~\cite{OED2013maths} sitt ursprung i grekiskans 
\emph{\ibygr{ma'qhma} (m{\'a}th{\=e}ma)}, vilket betyder \emph{lärande, 
studier, vetenskap}.
Det är i det antika Grekland som dagens matematik har sitt ursprung.
De studerade främst geometri och gjorde detta genom att sätta upp några 
grundläggande antaganden, kallade postulat eller axiom, som de var övertygade 
om att de stämde överens med verkligheten.
Dessa var enkla antaganden, såsom att två parallella linjer aldrig kommer att 
skära varandra.
Utifrån dessa enkla postulat härledde de olika geometriska resultat med hjälp 
av logik och de kunde bevisa att det måste vara på ett visst sätt.
Även om de kunde se genom några enkla experiment hur saker förhöll sig till 
varandra nöjde de sig inte utan ett bevis utifrån postulaten eller tidigare 
bevisade resultat.
Anledningen till detta är enkel: för att övertyga sig om att någonting alltid 
stämmer, då räcker det inte med att testa tre fall, det räcker inte med 
hundratals fall, inte en tusentals --- dessa fall kanske avviker, eller att 
inget av dem är det specialfall som avviker.

Detta har inspirerat matematiker genom historien och är den drivkraft som 
verkat för att matematiken utvecklats till det som den är idag.
Dagens matematik bygger likt grekernas på några enkla grundantaganden som vi 
kallar för axiom.
Vidare måste begrepp som vi använder definieras tydligt för att vi exakt ska 
veta vad som menas med dem.
Detta var drivkraften bakom axiomatiseringen av de naturliga talen som vi 
kommer att se i \cref{DeNaturligaTalen}, bakom grundläggningen av de hela 
talen i \cref{ch:Heltalen}, de rationella talen i 
\cref{ch:Rationella} och de reella talen i \cref{ch:Reella}.
Länge hade matematikerna tagit talen som självklara, men vid 1800-talets mitt 
behövde de veta tydligare vad ett tal var för att kunna gå vidare.
Det som är intressant är att behovet av en axiomatisering uppstod historiskt i 
omvänd ordning av hur de logiskt är uppbyggda och presenteras i denna text.
För en detaljerad redogörelse över den historiska utvecklingen hänvisas till 
\citet{Kline1990mtf3}.

I en definition av ett objekt eller egenskap sätter vi upp regler för hur ett 
objekt som är av denna typ eller har denna egenskap ska bete sig.
Om vi kan visa att ett objekt uppfyller reglerna i definitionen, då måste
objektet också vara av den typen eller ha den egenskapen.

\begin{example}\label{ex:cykel}
  Vi använder följande definition: en \emph{cykel} har åtminstone ett hjul och 
  vevarmar med pedaler som används för att driva hjulen och ge cykeln fart.

  Om vi tittar på fordonet i \cref{fig:cykel} ser vi att det har 
  åtminstone ett hjul (det har två), det har vevaxlar med pedaler som används 
  för att driva hjulen och således ge fordonet fart.
  Alltså måste det enligt vår definition vara en cykel.
\end{example}
\begin{figure}
  % XXX create figs/cykel.eps
  \includegraphics[width=0.7\textwidth]{figs/cykel.eps}
  \caption{En illustration av en cykel.}\label{fig:cykel}
\end{figure}

\begin{example}\label{ex:bil}
  Vi använder samma definition som i \cref{ex:cykel}.
  Om vi tittat på fordonet i figur \cref{fig:sparkcykel} ser vi att det 
  har åtminstone ett hjul.
  Det saknas dock vevarmar för att driva hjulen.
  Vi kan således konstatera att en sparkcykel enligt definitionen i 
  \cref{ex:cykel} inte är en cykel.
\end{example}
\begin{figure}
  % XXX create figs/sparkcykel.eps
  \includegraphics[width=0.7\textwidth]{figs/sparkcykel.eps}
  \caption{En illustration av en sparkcykel.}\label{fig:sparkcykel}
\end{figure}

Med denna typ av definitioner kan vi veta exakt, vi kan bevisa att ett objekt 
är av en specifik typ --- likt vi gjorde i exemplen ovan.
Vi kan också göra det omvända, om ett objekt är av denna typen uppfyller det de
givna reglerna.
Exempelvis, om någon säger att de har en cykel, då måste den ha åtminstone ett 
hjul och vevarmar med pedaler som driver den framåt.
När vi bevisar saker kan vi alltså utgå från enbart dessa regler, det är detta 
som är grunden inom matematiken.


%%%%%%%%%%%%%%%%%%%%%%%%%%%%%%%%%%
% NOTATION
%%%%%%%%%%%%%%%%%%%%%%%%%%%%%%%%%%
\section{Den missförstådda algebran}

Innan vi går vidare till våra studier av matematiken ska vi ägna några rader åt 
något som i onödan förrvirrat alltför många: algebra.
Algebran har många gånger beskrivits som \enquote{att räkna med bokstäver}, 
vilket är tämligen missvisande, vi ska i detta avsnitt förklara vad algebra 
faktiskt är.

I vårt dagliga tal har vi diverse språkliga konstruktioner för att veta vilka 
objekt vi samtalar om, exempelvis \enquote{den svarta bilen närmast hitåt} 
eller \enquote{korsningen närmast skolan}.
Vi använder dessa beskrivningar utan större ansträngning.
Ibland blir det dock fel, vi har säkert alla hamnat i en situation som krävt 
uttrycket \enquote{jaha, du menar den, jag trodde att du menade något helt 
annat}.
Då måste vi förbättra noggrannheten i våra beskrivningar av objekten för att 
bli förstådda utan vidare missförstånd.

Vi kan också prata om klasser av objekt, exempelvis träd.
De flesta har säkert stött på uttrycken \enquote{en björk har löv} eller 
\enquote{björkar har löv}, \enquote{en tall har barr} eller \enquote{tallar har 
barr}.
Det är för de flesta självklart vad som menas med en björk, att det är ett träd 
av släktet björk (\emph{betula}).
Den specifika arten spelar ingen roll, exempelvis hängbjörk (\emph{betula 
pendula}) och dvärgbjörk (\emph{betula nana}) är båda björkar.
Uttryckssättet brukar användas i meningar som \enquote{en björk har vit-svart 
stam} eller \enquote{björkar har vit-svarta stammar}.

\begin{exercise}
  Prata med någon om valfritt ämne, var uppmärksam på hur ni uttrycker er för 
  att säkerställa att ni \enquote{pratar om samma sak}.
\end{exercise}

Matematiken är inte särskilt mycket konstigare än vanligt språk.
Ett av matematikens främsta verktyg är just språket, utan det skulle vi inte 
kunna uttrycka våra tankar på ett effektivt sätt.
Ett annat ovärderligt verktyg är pennan med tillhörande papper.
Detta hjälper oss dels att spara våra tankar, men framförallt att strukturera 
dem.

Säg nu att vi vill diskutera olika exotiska blommor som finns hos en 
blomsterhandlare.
Då vi själva inte är experter på exotiska blommor vet vi inte namnet på dem, så 
vi kan inte använda namnen för att diskutera dem.
Ibland kanske vi släktets namn, exempelvis orkidé (\emph{orchidaceae}), men det 
finns så många olika orkidéer att detta inte alltid hjälper.
Om alla blommor är av olika färg skulle vi kunna använda färgen för att 
identifiera de olika blommorna, exempelvis \enquote{jag tycker bäst om den 
gröna blomman}.
Men, det är sällan detta är fallet, så det är inte en hållbar metod.
En annan naturlig metod är att numrera dem, exempelvis \enquote{jag tycker bäst 
om den första blomman vi tittade på}.
Denna metod är mer hållbar.

Nu vill vi göra samma sak med papper och penna, kanske för att göra en 
inköpslista.
Då det tar mycket längre tid att skriva något än att säga samma sak, vill vi 
naturligtvis ha en kortare notation.
Vi kanske bara skriver \enquote{1:a} eller bara \enquote{1}, istället för 
\enquote{den första blomman} eller \enquote{blomma nummer ett}.
Om vi nu vill ha flera av en sort, då finns det en risk att vi förvirrar oss om 
vi skriver \enquote{2 3}.
Betyder detta att vi vill ha två av den tredje blomman eller tre av den andra 
blomman?
En bättre lösning är därför att vi hittar på namn eller symboler till de olika 
blommorna.
Vi skulle kunna rita en liten bild av de blommor vi tycker bäst om och vill 
köpa.
Detta tar dock tid, och blir inte alltid särskilt bra.
Vi kan därför använda symboler som vi redan har namn på, och som vi kan skriva 
väldigt effektivt: vårt alfabet.
Låt den första blomman vara \(a\), den andra \(b\), och så vidare.
Då finns det ingen risk att vi blandar ihop antalet med sorten likt ovan, vi 
kan skriva exempelvis \(2c\) och vet exakt vad vi menar.

Detta är bakgrunden till den matematiska notationen med bokstäver.
Det är således inget märkligt med den matematiska notationen som skrämt så 
många, det är helt enkelt ett enkelt sätt att namnge saker på ett skrivvänligt 
sätt.
En nackdel är att det finns så få bokstäver.
Därför brukar man ibland även ta till det grekiska alfabetet för att få fler 
bokstäver.
En sammanfattning av detta finns i \cref{tbl:greekalpha}.

\begin{table}
  \caption{Det grekiska alfabetet.}
  \begin{tabular}{lll}
    \textbf{Versal} & \textbf{Gemen} & \textbf{Uttal} \\
    \toprule
    \(A\) & \(\alpha\) & alfa \\
    \(B\) & \(\beta\) & beta \\
    \(\Gamma\) & \(\gamma\) & gamma \\
    \(\Delta\) & \(\delta\) & delta \\
    \(E\) & \(\epsilon\) & epsilon \\
    \(Z\) & \(\zeta\) & zeta \\
    \(H\) & \(\eta\) & eta \\
    \(\Theta\) & \(\theta\) & theta \\
    \(I\) & \(\iota\) & iota \\
    \(K\) & \(\kappa\) & kappa \\
    \(\Lambda\) & \(\lambda\) & lambda \\
    \(M\) & \(\mu\) & my \\
    \bottomrule
  \end{tabular}
  \hspace{1em}
  \begin{tabular}{lll}
    \textbf{Versal} & \textbf{Gemen} & \textbf{Uttal} \\
    \toprule
    \(N\) & \(\nu\) & ny \\
    \(\Xi\) & \(\xi\) & xi \\
    \(O\) & \(o\) & omikron \\
    \(\Pi\) & \(\pi\) & pi \\
    \(P\) & \(\rho\) & rho \\
    \(\Sigma\) & \(\sigma\) & sigma \\
    \(T\) & \(\tau\) & tau \\
    \(Y\) & \(\upsilon\) & ypsilon \\
    \(\Phi\) & \(\phi\) & phi \\
    \(X\) & \(\chi\) & khi \\
    \(\Psi\) & \(\psi\) & psi \\
    \(\Omega\) & \(\omega\) & omega \\
    \bottomrule
  \end{tabular}\label{tbl:greekalpha}
\end{table}

Oftast brukar bokstäverna väljas så att de underlättar för minnet, exempelvis 
\(b\) för blomma och \(k\) för kruka.
Sedan kan man använda subskript, exempelvis kan vi beteckna de olika blommorna 
ovan som \(b_1, b_2\) och så vidare.
Då kan vi skilja blommorna från krukorna, som vi kan benämna \(k_1, k_2\) och 
så vidare.


%%%%%%%%%%%%%%%%%%%%%%%%%%%%%%%%%%%%%
% VAD ÄR MATEMATIK?
%%%%%%%%%%%%%%%%%%%%%%%%%%%%%%%%%%%%%
\section{Vad är matematik?}
% - Skillnaden mellan matematiken och andra vetenskaper.
% - Abstrakt konstruktion som endast finns i sinnet.
% - Naturvetenskap i Sverige, filosofi i resten av världen.
%   - Sveriges störta bidrag till matematikhistorien är att vi tog död på René
%     Descartes (Touraine, 1596--Stockholm, 1650).
% - Behöver inte ha någon till synes uppenbar tillämpning, det är vackert.
%   - Pierre de Fermat (160{1,7,8}--1665), advokat, upphovsman till Fermats
%     lilla sats och Fermats stora (sista) sats.
%   - Fermat--Euler satsen är grunden för kryptosystemet när man loggar in på
%     banken via internet.
% - Demokratiskt organiserad. Alla kan bidra, ex. Fermat var advokat.
Som antyddes ovan, kan matematiken beskrivas som studiet av abstrakta 
konstruktioner.
Med abstrakta konstruktioner menar vi saker som endast finns i våra tankar, 
eller vad Platon (cirka 428--348 f.v.t.) kallade \enquote{idévärlden}.
Vi sätter upp axiomen och definitionerna, vilka vi skulle kunna kalla våra 
spelregler, och undersöker sedan vad dessa spelregler ger upphov till.

Historiskt har matematiken ofta varit sammankopplad med studiet av
verkligheten.
Vi har kunnat studera verkligheten med hjälp av matematiken genom att våra
grundregler varit grundläggande principer för verkligheten\footnote{%
  Se exempelvis Euklides postulat för geometrin i \cref{ch:Geometri} eller~\cite[kap.\ 4]{Kline1990mtf1}.
}.
Men trots detta är matematiken skild från verkligheten.
De axiom vi utgår ifrån behöver inte vara principer från verkligheten.
Det finns matematiska konstruktioner som kan te sig så verklighetsfrånkopplade
att icke-matematiker ifrågasätter varför de studeras, och detta för oss in på
ett viktigt konstaterande:
Matematiken har inte alltid studerats enbart för att kunna dra slutsatser om
verkligheten.
Många matematiker genom historien studerade matematiken enbart för den rena
matematikens skull --- för att den var vacker, inte för att den gick att
tillämpa på verkligheten.
De ville utforska den värld som spänns upp av axiomen.
Exempel på sådana är Pierre de Fermat (cirka 1607--1650) som är upphovsman till 
den kända \emph{Fermats stora sats}.
Han var advokat och amatörmatematiker.
Fermats stora sats eller \emph{Fermats sista sats} säger att ekvationen
\(x^n+y^n=z^n\), där \(x,y\) och \(z\) är heltal, saknar lösningar för heltal
\(n\) som är större än två.
Fermat lämnade en anteckning i marginalen av sin kopia av Diophantus (omkring 
år 250) bok \emph{Arithmetica} att han hade ett bevis för detta, men att 
marginalen var för liten för att rymma det.
Det tog matematiker ända fram till år 1994 att bevisa satsen, så möjligen
hade Fermat inte ett korrekt bevis då beviset som togs fram inryms på tusentals 
sidor och krävde hundratals år av matematisk utveckling.
Han hade däremot ett korrekt bevis för sin \emph{lilla sats} som säger att om
\(p\) är ett primtal, då ger \(a^{p-1}\) alltid resten \(1\) vid division med
\(p\).
(Detta tas upp i \cref{ch:Talteori}.)
Leonard Euler (1707--1783) generaliserade Fermats lilla sats till att gälla även
sammansatta tal, och denna generalisering är känd som \emph{Eulers sats} eller 
ibland \emph{Fermat--Eulers sats}.
Resultaten för dessa hade inget tillämpningsvärde för tiden, utan drivkraften
var att utforska matematikens vackra värld och finna vackra resultat som dessa.
År 1978 publicerades dock en tillämpning av satsen.
Det var Ronald Rivest (1947--), Adi Shamir (1952--) och Leonard Adleman 
(1945--) som då publicerade ett kryptosystem sedermera känt som RSA\@.
RSA-systemet bygger på Eulers sats och systemet ligger till grund för
mycket av den säkra kommunikationen som sker på internet idag.
Det dröjde alltså cirka 300 år innan någon fann en tillämpning, innan dess var 
det vara ett vackert resultat.

Detta visar även vikten av den så kallade grundforskningen, den forskning som 
inte har någon omedelbar tillämplighet, utan enbart syftar till att fördjupa 
mänsklighetens kunskap inom området.
Detta är inte viktigt bara för matematiken utan för alla vetenskaper.
För vi kan ställa oss frågan: hade vi hade haft säker kommunikation på internet
idag om vi bara utforskat det som verkat direkt tillämpbart?

%Vi ska med det gå vidare till nästa kapitel som handlar om matematikens grund
%-- logik och bevis.
%Det är logiken som är matematikens verktyg för att resonera kring de axiom som
%vi antagit.


%%%%%%%%%%%%%%%%%%%%%%%%%%%%%%%%%
% UPPLÄGG
%%%%%%%%%%%%%%%%%%%%%%%%%%%%%%%%%
%\section{Bokens upplägg}
%...
%
