\chapter{Grundläggande kombinatorik}
\label{ch:Kombinatorik}
% XXX Skriv kapitlet om kombinatorik

\begin{theorem}[Dirichlets lådprincip]\label{thm:ladprincipen}
  % XXX Beskriv Dirichlets lådprincip
  \dots
\end{theorem}

\begin{theorem}[Additionsprincipen]
  % XXX Beskriv additions principen
  \dots
\end{theorem}

\section{Om val}
Vi stöter ofta på val av olika slag.
Vi ska i detta avsnitt reflektera över antalet möjliga utfall som kan uppstå
genom att olika val kombineras samman.

Vi inleder med definitioner av de begrepp vi kommer att använda.

\begin{definition}\label{def:Val}
  Med \emph{val} menas att det finns \(n \geq 1\) antal distinkta alternativ
  att välja mellan, av dessa alternativ måste ett och endast ett väljas.
  Det alternativ som väljs sägs vara \emph{utfallet av valet}.
\end{definition}

Vi fortsätter med ett väldigt enkelt lemma om förhållandet mellan valets
antal alternativ och det möjliga antalet utfall.

\begin{lemma}\label{lem:AntalUtfall}
  Ett val från \(n\) alternativ har \(n\) möjliga utfall.
\end{lemma}
\begin{proof}
  För varje alternativ behövs åtminstone ett utfall.
  Om vi har \(n\) alternativ och antar att vi har \(n+1\) utfall,
  då måste det enligt Dirichlets lådprincip, \cref{thm:ladprincipen}, vara 
  något alternativ som får fler än ett utfall.
  Men om ett och samma alternativ har flera utfall, då måste dessa utfall
  vara samma utfall.
  Detta är en motsägelse och därför måste vi ha exakt \(n\) utfall.
\end{proof}

För fullständighet inkluderas även följande exempel.
\begin{example}\label{ex:Fikabrod}
  Du ska välja fikabröd till eftermiddagsfikat.
  De alternativ du har att välja mellan är en nybakad bulle, en torr kaka och
  att inte ta något fikabröd.
  Notera att välja \emph{ingenting} utgör ett alternativ, det går alltså inte
  att avstå från ett val.

  Valet av fikabröd har tre alternativ, enligt \cref{lem:AntalUtfall}
  finns således tre möjliga utfall.
  Ett utfall är att vi väljer bullen, ett annat att vi väljer kakan och det
  sista att vi väljer att inte ta något fikabröd.
\end{example}

\subsection{Att välja bland val}
Då är det dags att utöka våra möjligheter att välja genom att kombinera flera
val till ett sammansatt val.

\begin{definition}
  När ett val ska göras följt efter ett annat säger vi att vi har ett
  \emph{sammansatt val}.
  Ett sammansatt val kan ibland kallas för val.
  Varje ingående val kallas för ett \emph{delval}.
\end{definition}

\begin{example}
  Det är dags för eftermiddagsfika igen.
  Du ska först välja om du ska dricka kaffe, te eller vatten.
  (Att inte välja någonting är inte ett alternativ i detta val, detta är ett
  rent teoretiskt val eftersom att rent praktiskt finns ju alltid
  alternativet att inte fika alls.)
  Därefter ska du välja fikabröd enligt \cref{ex:Fikabrod}.
  Då har vi ett sammansatt val bestående av två delval, ett för dryck och ett
  för fikabröd.
\end{example}
\begin{exercise}
  Visa att detta stämmer överens med att välja mellan alternativen \emph{kaffe 
  och bulle}, \emph{kaffe och kaka}, \emph{te och bulle}, och så vidare.
\end{exercise}

Det är nu intressant att veta hur många möjliga utfall som ett sådant val
möjligen kan ha.
Detta sammanfattas i följande sats.

\begin{theorem}[Multiplikationsprincipen]\label{thm:Multiplikationsprincipen}
  Ett sammansatt val av \(m\) antal delval, där delval \(i\) har \(n_i\)
  antal alternativ,
  har \(n_1\cdot n_2\cdots n_m\) antal möjliga utfall.
\end{theorem}

\begin{proof}
  Låt oss börja med att titta på det sista valet, val \(m\).
  Detta val har \(n_m\) alternativ och således \(n_m\) utfall enligt
  \cref{lem:AntalUtfall}.
  Vi går vidare till valet innan, det vill säga val \(m-1\).
  Detta val har \(n_{m-1}\) möjliga utfall.
  För varje utfall av detta val kan vi få \(n_m\) utfall i val \(m\).
  Då har vi alltså tillsammans
  \begin{equation}\label{eq:MultprincipTvaVal}
    \sum_{k=1}^{n_{m-1}} n_m = \underbrace{n_m + n_m + \cdots
    n_m}_{n_{m-1}} = n_{m-1}\cdot n_m.
  \end{equation}
  För varje utfall av delval \(m-2\) kan vi få antalet utfall från
  \cref{eq:MultprincipTvaVal}.
  Det vill säga
  \begin{equation}
    \sum_{k=1}^{n_{m-2}} n_{m-1}\cdot n_m = n_{m-2}\cdot n_{m-1} \cdot n_m.
  \end{equation}
  Vi fortsätter på detta vis till vi når det första valet då vi får
  \begin{equation}
    \sum_{k=1}^{n_1} n_2\cdot n_3\cdots n_{m-2}\cdot n_{m-1}\cdot n_m
      = n_1\cdot n_2\cdots n_m,
  \end{equation}
  vilket visar satsen.
\end{proof}

Från \cref{thm:Multiplikationsprincipen} följer direkt ett enkelt resultat
som vi ger i detta korollarium\footnote{%
  Det vill säga en följdsats.
}.

\begin{corollary}\label{cor:SammansattValKonstAlternativ}
  Ett sammansatt val av \(m\) antal delval där varje delval har \(n\)
  alternativ, har \(n^m\) antal möjliga utfall.
\end{corollary}

\begin{proof}
  Om vi har ett sammansatt val av \(m\) antal delval, där delval
  \(i\) har \(n_i\) alternativ.
  Då har vi enligt \cref{thm:Multiplikationsprincipen} att det totala
  utfallet är \(n_1\cdot n_2\cdots n_m\).
  Men eftersom att alla val hade samma antal alternativ, nämligen \(n\), då
  får vi att \(n_1\cdot n_2\cdots n_m = n\cdot n\cdots n = n^m\).
  Följaktligen får vi \(n^m\) antal möjliga utfall då vi har \(m\) delval där
  varje delval har \(n\) alternativ.
\end{proof}



\section{Val av lösenord}
Vi ska nu titta på hur detta kan användas för att undersöka säkerheten hos
lösenord.
Det finns flera angreppssätt för att skapa lösenord, exempelvis genom att
välja en kombination av tecken (bokstäver, siffror och specialtecken) eller att
helt enkelt välja några ord.
Det går också att skapa lösenord genom att slumpmässigt välja några ord som
kombineras till ett lösenord.

Vi börjar med det första fallet, där vi skapar ett lösenord genom att kombinera
tecken.
Om vi ska skapa ett lösenord som är fem tecken långt och får innehålla
bokstäverna A-Z, både gemener och versaler,
siffrorna 0-9 samt
specialtecknen ''!@.\%\&'',
då kan vi se valet av lösenord som ett sammansatt val bestående av fem delval,
där alternativen för varje delval är de tillåtna tecknen.
Antalet alternativ är således \(26\cdot 2 + 10 + 5 = 67\).
Eftersom att alla delval har samma antal alternativ kan vi använda
\cref{cor:SammansattValKonstAlternativ} för att ta reda på antalet möjliga
utfall, det vill säga antalet möjliga lösenord.
Antalet lösenord som uppfyller detta krav är \(67^5 = 1350125107\).

Nu fortsätter vi med att titta på fallet med att välja lösenord genom att
slumpmässigt välja några ord.
Vi bestämmer oss för att använda ett lösenord med fyra slumpmässigt valda
ord från svenska språket.
Enligt Svenska Akademin \citep{SAOL} innehåller Svenska Akademins Ordlista
ungefär 125000 ord.
Detta ger enligt \cref{cor:SammansattValKonstAlternativ}
\begin{equation}\label{eq:AntalUtfallOrd}
  125000^4 = (2^3\cdot 5^6)^4
    = 2^{12}\cdot 5^{24}
    = 2^{12}\cdot 25^{12},
\end{equation}
det vill säga \(244140625000000000000\), möjliga lösenord.

Det senaste fallet kan också ses ur det första fallets perspektiv.
Låt oss anta att medellängden av orden vi väljer från är fem tecken och att
dessa tecken enbart är gemener.
Det innebär att vi får
\begin{equation}\label{eq:AntalUtfallTecken}
  (29^5)^4 = 29^{20} = 176994576151109753197786640401
\end{equation}
möjliga lösenord.

Hur spelar denna representation någon roll, vad betyder skillnaden mellan
\cref{eq:AntalUtfallOrd} och \cref{eq:AntalUtfallTecken}?
Det första som bör påpekas är att i \cref{eq:AntalUtfallTecken} tas även
teckenkombinationer som ej är svenska ord med.
Detta eftersom att valet var att välja fyra uppsättningar av fem tecken.
Betydelsen av detta är att om vi låter en dator bara slumpa fram 20 bokstäver
(gemener), då kommer det att resultera i antalet gissningar från
\cref{eq:AntalUtfallTecken}.
Det är inte ens säkert att datorn kommer att hitta rätt lösenord om vi råkade
välja fyra långa ord som alla var minst sex bokstäver långa.
Detta är ett problem med denna uppskattning.
Om vi däremot har en ordlista som datorn väljer ord från för att kombinera
dessa till ett lösenord om fyra ord för att gissa lösenordet, då kommer antalet
gissningar att maximalt bli \cref{eq:AntalUtfallOrd}.
Även denna metod kräver naturligtvis att orden i lösenordet finns med i
ordlistan som datorn har tillgång till.

