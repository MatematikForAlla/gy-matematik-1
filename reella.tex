\chapter{De reella talen}
\label{ch:Reella}%\nocite{Kline1990mtf3,KTHCirkel2005rt}
% XXX Skriv om de reella talens konstruktion
\dots

%%%%%%%%%%%%%%%%%%%%%%%%%%%%%%%%%%%%%
% DEDEKINDS SNITT
%%%%%%%%%%%%%%%%%%%%%%%%%%%%%%%%%%%%%
\section{Dedekinds snitt}
% XXX Skriv om Dedekinds snitt
\dots

\begin{definition}[Snitt]
  % XXX Definiera Dedekinds snitt
  \dots
\end{definition}


%%%%%%%%%%%%%%%%%%%%%%%%%%%%%%%%%%%%%
% ARITMETIK
%%%%%%%%%%%%%%%%%%%%%%%%%%%%%%%%%%%%%
\section{Aritmetik}
% XXX Skriv om aritmetik med de reella talen
\dots

\begin{definition}[Algebraiska egenskaper hos de reella talen]
  På mängden \(\R\) definieras två binära operatorer, addition (\(+\)) och
  multiplikation (\(\cdot\)).
  För addition gäller följande:
  \begin{description}
    \item[Kommutativitet] \(a+b=b+a\) för alla \(a,b\in\R\).
    \item[Associtivitet] \((a+b)+c=a+(b+c)\) för alla \(a,b,c\in\R\).
    \item[Additiv enhet] Det finns ett element \(0\in\R\) sådant att
      för alla \(a\in\R\) gäller att \(0+a = a+0 = a\).
    \item[Additiv invers] För alla \(a\in\R\) finns ett element \(-a\in\R\)
      sådant att \(a + (-a) = (-a) + a = 0\).
  \end{description}
  För multiplikation gäller följande:
  \begin{description}
    \item[Kommutativitet] \(a \cdot b=b \cdot a\) för alla \(a,b\in\R\).
    \item[Associtivitet] \((a \cdot b) \cdot c=a \cdot (b \cdot c)\) för
      alla \(a,b,c\in\R\).
    \item[Multiplikativ enhet] Det finns ett element \(1\in\R\) sådant att
      för alla \(a\in\R\) gäller att \(1 \cdot a = a \cdot 1 = a\).
    \item[Multiplikativ invers] För alla \(0 \neq a\in\R\) finns ett
      element \(1/a\in\R\) sådant att
      \(a \cdot (1/a) = (1/a) \cdot a = 1\).
  \end{description}
  Utöver detta gäller även
  \begin{description}
    \item[Multiplikativ distribuitet över addition]
      \(a \cdot (b+c) = (a \cdot b) + (a \cdot c)\) och
      \((b+c) \cdot a = (b \cdot a) + (c \cdot a)\) för alla reella tal
      \(a,b,c\in\R\).
  \end{description}
\end{definition}

\begin{exercise}
  Utforska vad de algebraiska egenskaperna hos de reella talen tillåter.
\end{exercise}
\begin{exercise}
  Om \(x\in\R\) och \(a\in\R\) båda är reella tal och \(x + a = a\),
  visa att \(x=0\).
\end{exercise}
\begin{exercise}
  Om \(x\in\R\) och \(a\in\R\) båda är reella tal och \(a \cdot x = a\),
  visa att \(x=1\).
\end{exercise}
\begin{exercise}
  Om \(a\in\R\) är ett reellt tal, visa att \(a \cdot 0 = 0\).
\end{exercise}
\begin{exercise}
  Om \(0 \neq a \in \R\) och \(b\in\R\) är reella tal och \(a \cdot b = 1\),
  visa att \(b = 1/a\).
\end{exercise}
\begin{exercise}
  Om \(a\in\R\) och \(b\in\R\) är reella tal och \(a \cdot b = 0\),
  visa att antingen \(a=0\) eller \(b=0\), eller båda.
\end{exercise}


%%%%%%%%%%%%%%%%%%%%%%%%%%%%%%%%%%%%%%%%%%
% POTENSER
%%%%%%%%%%%%%%%%%%%%%%%%%%%%%%%%%%%%%%%%%%
\subsection{Potenser}
% XXX Skriv om potenser med de reella talen
\dots

